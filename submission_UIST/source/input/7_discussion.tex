\section{Discussion}
With this paper, we contributed a new way to design the experience of action augmentation with the potential to maintain a user's SoA: A BCI to control EMS at moments of users' intention to interact. By leveraging an average of 11 EEG channels and using a simple, fast-to-compute feature, the BCI achieved a mean F1 score of about .7. We demonstrated in ten participants that one effect of SoA, intentional binding, is maintained when using our system. However, the subjectively rated level of control was lower when using the system as compared to voluntary interaction without EMS. This was differentiated by participants' qualitative reports: when the stimulation aligned with participants' intentions, positive sentiments outweighed negative ones. When it did not, participants reported a negatively perceived loss of control.

Our system was built using off-the-shelf, affordable, equipment to physically augment users' actions. Taken together, all technical devices to control the users' movements, i.e., the EMS stimulation device, Arduino, and switchboard, cost less than 100 Euros. While we used a 64-channel research-grade EEG system, the channel selection procedure resulted in the system ultimately using a low-density channel coverage for classification. Today, many low-density EEG devices are available on consumer markets at affordable price points, see~\citet{Niso2023-ce} for a recent summary of available wireless systems. 

Still, the system achieved what is considered to be a `good' F1 score of .7, and hence sometimes detected users' intention to interact. In our user study, this system performance affected users' tendency for temporal binding when comparing the three experimental conditions. 

We first conclude that augmentation generally impacted SoA as was evident in the users' sentiments and ratings of control. While we believe that our prototype did not perform at a consistently high enough performance to allow for more fine-grained inferences about the the pre-emptive gain it achieved, we maintain that in some cases it elicited an pre-emptiot that maintained agency to a certain degree. We believe this to be a very promising finding because our prototype with low technological requirements still produces behavioral results indicating that intention can be detected and planned movements be augmented.

Our system ran at a 10 Hz update rate. This was chosen in order to avoid buffer overflow due to the fact that the end-to-end computing latency from measuring the physiological signal to outputting the predicted class and probability took about 40 ms. The RP is reported to occur several hundred milliseconds prior to the physical movement~\cite{Schurger2021-vp}. Therefore, our augmentation should have resulted in a motion preemption, considering subtracting the computing latency and a delay that the EMS stimulation takes to affect the muscle from the RP-based intention detection. However, it is difficult to assess whether participants' movements were in fact pre-empted as no label exists for the `true' intention onset to compare the classifier to. That participants were able to act on their own volition, and hence no labelling of their movement intention could be derived, is the key distinction to related works using controlled stimulus-response paradigms.

However, we learned from these paradigms that increasing participants' reaction times by about 80 ms is optimal with regards to maintaining SoA~\cite{Kasahara2019-sk}. On one hand, our system is capable of delivering pre-emption within this `SoA-optimal' time range. On the other hand, however, the sub-optimal performance of the classifier resulted in a lot of variation due to false positive detections. While in some cases, as we observed in the users' comments, the stimulation may have pre-empted a user's motion, false positive stimulation may have had a significant impact on the overall impression of the system, see section~\ref{improving_classifier} below.

In line with the literature, we found that participants estimated the delay between tap and tone to be lower in both conditions where they were instructed to act on their own volition as compared to the passive condition~\cite{Moore2012-dk}. We point out that recent literature has questioned the validity of the intentional binding phenomenon as a correlate of agency, stating that the effect may ``merely represent a strong case of multisensory causal binding.''~\cite{Suzuki2019-pi, Gutzeit2023-ei, Hoerl2020-my}. However, in our study, we observed the effect of compressing the action-outcome delay to be maintained even when using the system. Hence, the effect was maintained for both conditions of voluntary action.

% Bad -> (1) situate the findings of the IB, (2) bad classification metrics not good enough to feel in control
\subsection{Limitations}
Two main procedural issues arose concerning the time estimation task: First, contrasting AUGMENTED with INTENTION, we noted that when EMS moved participants' fingers, they sometimes reported that their finger was pressing on the touchscreen for a longer duration as compared to their `normal' touchscreen tap. This may have introduced additional variance in the time estimation task since the exact moment of the tap that causes the tone is more obscure. Second, following pilot recordings, we chose to obtain participants' time estimates by asking them to type in their estimates on the keyboard. However, we observed that many participants, while perceiving a continuous distribution of the delays, did not answer at a continuous ms resolution but rather at steps of 50 or 100 ms, skewing the distribution of their answers. While many different versions of the intentional binding paradigm exist~\cite{Moore2012-dk}, we would choose a continuous slider for future experiments with different initial positions over trials to reduce the bias in participants' estimates.

\subsubsection{Improving the Prediction of the Intent to (Inter-) act}\label{improving_classifier}
While an F1 score of .7 may be considered `good' for many classification scenarios, when aiming to elicit a feeling of control this level of performance may likely not be good enough, see~\citet{Papenmeier2022-oi} for a review. Balancing false positive rate (FPR) and true positive rate (TPR) appropriately may prove crucial to elicit an experience of agency.

For the current study, we must note the following limitations arising from the (potentially) insufficient performance of the classification system. To provide a full picture we first describe the impact from a signal detection point of view: False negative classifications meant that the system did not trigger an EMS pulse in line with participants intent to tap on the screen. On the other hand, false positive classification meant that an EMS pulse was sent in the absence of a true intention by the participant. Both cases, individually and jointly, had the potential to impact the user experience and erode trust in the system. In the interviews, participants reported only a relatively low number of correctly detected intentions. This may have ultimately led to a misalignment between users' perceptions and the observed, measurable, system behavior. Furthermore, the necessity for a fixed block/condition order may have further contributed to this effect. However, keeping the order was necessary in order to first obtain training data for the BCI based on the unique EEG signals of each participant.

As a consequence, the insights gained into the system's potential to preserve a sense of agency are limited. The quantitative metrics employed in our study, specifically those measuring the sense of control (including temporal binding and item assessment), may not have captured the nuanced temporal alignment experiences associated with EMS augmentation and user movement intentions but rather a more holistic assessment of the entire experimental block, which included trials with missed stimulations, too early stimulations, and some trials where the stimulation was in alignment with the participants' intent.

% Good -> This can be improved by using Gaze Dynamics, EMG and other things
% If it would work, how can it be used? Implications -> playful experiences, adaptive interfaces
Hence, a \textit{key} objective remains in improving the overall performance of the classification system. This can be accomplished by fusing several complementary models. For virtual reality (VR) platforms,~\citet{David-John2021-vg} have recently shown that gaze dynamics, especially gaze velocity, carry information with respect to the intent to interact. Furthermore,~\citet{Nguyen2023-me} presented further evidence that physiological signals originating from muscle activity (EMG) provide a very reliable label of movement onset~\cite{Nguyen2023-me}. Arguably, an augmentation system based on several models tuned to specific physiological features will reach sufficient classification performance in the very near future.

% Conclusion
\section{Conclusions}
In this paper, we demonstrated a system to maintain users' SoA during augmented experiences using brain signals reflecting the intent to (inter-) act. In our user study, we found that intentional binding effects were stronger when participants worked with the system to tap on a screen compared to when they were passively moved. Furthermore, the level of control working with the system was higher than when being passively moved, as indicated by subjective responses.

We believe this to be an important next step towards augmented users with full integration of the technology~\cite{Mueller2020-dl}. Ultimately, our closed-loop system enables novel predictive interfaces that leverage the users' body directly yet still feel \textit{natural} as they align with the users' intention: For example, altering the affordance structure of the interaction in real-time~\cite{Gehrke2022-kz, Lopes2015-ze, Nataraj2020-wm}.

With AI technologies becoming more and more ubiquitous in our everyday lives, questions about how agency is shared between us humans and AI technologies arise regularly. With the likely future that these systems move directly onto our bodies, alignment with users' intentions will be the \textit{key} component driving their adoption.

