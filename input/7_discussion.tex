\section{Discussion}
In this paper, we investigated if SoA can be maintained in action augmentation when the augmentation aligns with the users intention. We evaluated a BCI that controls EMS at moments of users' intention to interact. By leveraging an average of 11 EEG channels and using a simple, fast-to-compute feature, the BCI achieved a mean F1 score of about .7. 

%% I Agency measures: 

% 1. IB Results
In the user study with 10 participants we found no evidence for a disruption in intentional binding, hinting at a maintained SoA. In line with the literature, participants underestimated the delay between tap and tone in both conditions where they were instructed to act on their own volition~\cite{Moore2012-dk}. However, we also found that participants similarly underestimated the delay in INVOLUNTARY. This conflicting finding is in line with recent literature that has questioned the validity of the intentional binding phenomenon as a correlate of agency, stating that the effect may ``merely represent a strong case of multisensory causal binding.''~\cite{Suzuki2019-pi, Gutzeit2023-ei, Hoerl2020-my}. This might be especially true for cases where one's own body is completing the action while being externally controlled. Interestingly, in our scenario, proprioceptive signals and other indicators of embodied actions remain in line with the user acting at their own volition. Hence, the underestimations we found might purely follow from multisensory causal binding. The temporal compression effect might therefore not be based on inferring subjective causality following from intention but solely from causally binding sensory information. 

We do note that the underestimation appeared to be trending smaller in INVOLUNTARY as in both, INTENTION and AUGMENTED. However, the effect might be smaller than what could be proven given our sample size. Within the AUGMENTED condition trials, we observed no difference between `EMS-controlled' --and `muscle-controlled' trials, indicating an alignment with the augmentation technology in the AUGMENTED's `EMS-controlled' trials as the stimulation did not appear to disrupt participants' performance in the temporal delay estimation task.

% 2. Questionnaire & Interviews
On the level of subjective experience, we found that participants rated their level of control lower when using the system (AUGMENTED) as compared to voluntary interaction without EMS (INTENTION). This is a common finding when EMS is used for muscle control. For example,~\citet{Lopes2015-dk} reported people frequently not feeling in control in an interaction facilitated via EMS. INTENTION than might serve a very high mark for ratings following AUGMENTED and a better fitting baseline comparison would make sense. One way to better assess agency attributions when working with the system could be a comparison to earlier works that also leveraged EMS instead of our INTENTION condition where no EMS was in use. Triggering EMS based on simple heuristics (or sham) and contrasting this with BCI-controlled EMS stimulation could shed light on agency attributions exclusively explained via the BCI control. As a next step forward, stimulus-response paradigms like those of~\citet{Kasahara2019-sk, Kasahara2021-gy} could then serve as the next step-up in evaluation, with the augmentation performance, such as \textit{pre-emptive} gain, coming into focus.

%% Now use subjective reports and comments to go over into system performance and evaluation of false positive using prediction error ERPS
From participants' qualitative reports we learned that when the stimulation aligned with participants' intentions, positive sentiments outweighed negative ones. When it did not, participants reported a negatively perceived loss of control. To situate participants comments, a differentiated look at the classification system's performance is warranted.

%% II System performance and prediction error ERPs to obtain labels
First, we note that our system was built using off-the-shelf, affordable, equipment to physically augment users' actions. Taken together, all technical devices to control the users' movements, i.e., the EMS stimulation device, Arduino, and switchboard, cost less than 100 Euros. While we used a 64-channel research-grade EEG system, the channel selection procedure resulted in the system ultimately using a low-density channel coverage for classification. Today, many low-density EEG devices are available on consumer markets at affordable price points, see~\citet{Niso2023-ce} for a recent summary of available wireless systems. Still, the system achieved what is considered to be a `good' F1 score of .7, and hence sometimes detected users' intention to interact.

While an F1 score of .7 may be considered `good' for many classification scenarios, when aiming to elicit a feeling of control this level of performance may likely not be good enough, see~\citet{Papenmeier2022-oi} for a review. Balancing false positive rate (FPR) and true positive rate (TPR) appropriately may prove crucial to elicit an experience of agency.

\subsection{ERPs for Data Labeling}
% To provide the full picture we first describe the impact from a signal detection theory point of view: 
False negative classifications meant that the system did not trigger an EMS pulse in line with participants' intent to tap on the screen. On the other hand, false positive classification meant that an EMS pulse was sent in the absence of a true intention by the participant. Both cases, individually and jointly, had the potential to impact the user experience and erode trust in the system. 

We investigated EEG activity at electrode FCz to ascertain presence or absence of a prediction error in response to the stimulation or movement onset, see figure~\ref{fig:erp}. Ultimately, classifying the EEG here could be leveraged to approximate labels for classifier validation. The idea is that in AUGMENTED a false positive would mean that the EMS is falsely triggered and hence the movement onset would catch the participant by surprise. On the other hand a false negative means that a user starts their movement on their own without EMS, potentially equally `surprising' them, or in other words not aligning with their prediction. In line with this theory, we found a negativity affected by the trial condition in the  150--250 ms time window after movement onset, see figure~\ref{fig:erp}a. A strong negativity was present in INVOLUNTARY, where the user experienced the stimulation at unpredictable times. A less pronounced negativity was present in AUGMENTED, with only a marginal peak present in INTENTION.

To carve out the differences between the two trial categories `EMS-controlled', and `muscle-controlled' in AUGMENTED from the two \textit{anchoring} conditions, we subtracted each category from INVOLUNTARY and INTENTION accordingly, see figure~\ref{fig:erp}b for an explorative view on the ERP data.

First, we note that all `muscle-controlled' trials from AUGMENTED were false negatives, as the classifier failed to pick up on the readiness potential preceding the movement. By subtracting AUGMENTED, we observed that the pre-movement activity at electrode FCz did not differ from INTENTION. This shows that both these trial groupings exhibited a similar readiness potential at electrode FCz and the classifier failed to pick up on on it. Following the movement start, a trend emerged after 150--250 ms, in which `muscle-controlled' trials from AUGMENTED trends towards a stronger negativity than INTENTION trials, hence a positive difference, see figure~\ref{fig:erp}b blue line. This may indicate that participants always expected an EMS pulse in AUGMENTED and carrying out the movement without EMS \textit{support} violated their prediction. 

In the `EMS-controlled' trials in AUGMENTED, both false positive and true positive classifications overlap. In the contrast with INVOLUNTARY, the readiness potential which lead towards a positive classification outcome was visible in the second preceding the movement, see the negative going deflection in figure~\ref{fig:erp}b orange line. In the 150--250 ms time window after movement onset, a similar trend as described above was visible. The slight positive bump was due to a stronger negativity in `EMS-controlled' trials in AUGMENTED as compared to INVOLUNTARY. One possible explanation is that the trend is driven by false positive classifications, where a slight misalignment of the stimulation, i.e. it being too early, severely disrupts the user, eliciting a string prediction error signal. Taken together, we believe that contrasting and classifying single-trials could be a fruitful endeavor for approximating classification labels.

\subsection{Limitations \& Future Directions}
Two main procedural issues arose concerning the time estimation task: First, contrasting AUGMENTED with INTENTION, we noted that when EMS moved participants' fingers, they sometimes reported that their finger was pressing on the touchscreen for a longer duration as compared to their `normal' touchscreen tap. This may have introduced additional variance in the time estimation task since the exact moment of the tap that causes the tone is more obscure. Second, following pilot recordings, we chose to obtain participants' time estimates by asking them to type in their estimates on the keyboard. However, we observed that many participants, while perceiving a continuous distribution of the delays, did not answer at a continuous ms resolution but rather at steps of 50 or 100 ms, skewing the distribution of their answers. While many different versions of the intentional binding paradigm exist~\cite{Moore2012-dk}, we would choose a continuous slider for future experiments with different initial positions over trials to reduce the bias in participants' estimates.

\subsubsection{System Performance}
In the interviews, participants reported only a relatively low number of correctly detected intentions. This may have ultimately led to a misalignment between users' perceptions and the observed, measurable, system behavior. Furthermore, the necessity for a fixed block/condition order may have further contributed to this effect. However, keeping the order was necessary in order to first obtain training data for the BCI based on the unique EEG signals of each participant.

As a consequence, the insights gained into the system's potential to preserve a sense of agency are limited. The quantitative metrics employed in our study, specifically those measuring the sense of control (including temporal binding and item assessment), may not have captured the nuanced temporal alignment experiences associated with EMS augmentation and user movement intentions but rather a more holistic assessment of the entire experimental block, which included trials with missed stimulations, too early stimulations, and some trials where the stimulation was in alignment with the participants' intent.

Our system ran at a 10 Hz update rate. This was chosen in order to avoid buffer overflow due to the fact that the end-to-end computing latency from measuring the physiological signal to outputting the predicted class and probability took about 40 ms. The RP is reported to occur several hundred milliseconds prior to the physical movement~\cite{Schurger2021-vp}. Our augmentation then should have resulted in a motion pre-emption, considering a subtraction of the computing latency and a delay that the EMS stimulation takes to affect the muscle from the RP-based intention detection. However, it is difficult to assess whether participants' movements were in fact pre-empted as no readily available label exists for the `true' intention onset to compare the classifier to. That participants were able to act on their own volition, and hence no direct label of their movement intention could be derived, is the key distinction to related works using controlled stimulus-response paradigms.

While we believe that our prototype did not perform at a consistently high enough performance to allow for more fine-grained inferences about the the pre-emptive gain it achieved, we maintain that in some cases it elicited a pre-emption that maintained agency to a certain degree. We believe this to be a very promising finding because our prototype with low technological requirements still produces behavioral results indicating that intention can be detected and planned movements be augmented. However, from stimulus-response paradigms we recall that increasing participants' reaction times by about 80 ms is optimal with regards to maintaining SoA~\cite{Kasahara2019-sk}. On one hand, our system is capable of delivering pre-emption within this `SoA-optimal' time range. On the other hand, however, the sub-optimal performance of the classifier resulted in a lot of variation due to false positive detections. While in some cases, as we observed in the users' comments, the stimulation may have pre-empted a user's motion, false positive stimulation may have had a significant impact on the overall impression of the system.

Taken together, the \textit{key} objective remains in improving the overall performance of the classification system. This can be accomplished by fusing several complementary models. For virtual reality (VR) platforms,~\citet{David-John2021-vg} have recently shown that gaze dynamics, especially gaze velocity, carry information with respect to the intent to interact. Furthermore,~\citet{Nguyen2023-me} presented further evidence that physiological signals originating from muscle activity (EMG) provide a very reliable label of movement onset~\cite{Nguyen2023-me}. Arguably, an augmentation system based on several models tuned to specific physiological features will reach sufficient classification performance in the very near future.

% Maybe add something here? If it would work, how can it be used? Implications -> playful experiences, adaptive interfaces

\section{Conclusions}
In this paper, we designed and investigated a system to maintain users' SoA during augmented experiences using brain signals reflecting the intent to (inter-) act. In our user study, we found no convincing evidence that intentional binding effects are stronger when participants work with an augmentation system compared to being passively moved. However, participants rated their level of control working with the system higher than when being passively moved.

We believe this to be an important next step towards augmented users with full integration of the technology~\cite{Mueller2020-dl}. Ultimately, our closed-loop system is another step towards novel predictive interfaces that leverage the users' body directly yet still feel \textit{natural} as they align with the users' intention. For example, we envision that such closed-loop stimulation systems will prove useful in altering the affordance structure of an interaction in real-time~\cite{Gehrke2022-kz, Lopes2015-ze, Nataraj2020-wm}.

With AI technologies becoming more and more ubiquitous in our everyday lives, questions about how agency is shared between us humans and AI technologies arise regularly. With the likely future that these systems move directly onto our bodies, alignment with users' intentions will be the \textit{key} component driving their adoption.