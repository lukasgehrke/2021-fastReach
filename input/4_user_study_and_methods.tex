
\section{Methods}

With this study we wanted to find out whether agency can be preserved during \textit{physical actuation} of the user's body when their own brain signals serve as the control signal. To this end, we build a system that leverages the brain signals known to reflect the volitional intent to interact, i.e. the EEG readiness potential, to trigger electrical muscle stimulation to cue a reaching movement.

To assess if agency was preserved, we used a mixed methods approach including a psychometric test of intentional binding, questionnaires as well as qualitative interviews. Our interviews followed a phenomenological structure \todo{add reference from AH review paper mueller}. We t

\subsection{Participants}

\todo{add actual number} Ten participants (\todo{add mean age and sd}) from our local institution participated in our study. All participants were right handed, had normal or corrected to normal vision and had not experienced EMS before. Participants were compensated with 12 Euro per hour of study participation. Prior to their participation, participants were informed of the nature of the experiment, recording and anonymization procedures and signed a consent form.

\subsection{Apparatus}
The experimental setup, depicted in Figure~\ref{}, comprised: (1) a wrist-mounted wearable HTC VIVE tracker for motion capture, (2) a 64-channel EEG system, and (3) a medically-compliant EMS device connected via two electrodes worn on the forearm. To assist readers in replicating our experiment, we provide the necessary technical details, the complete source code to the experiment, the collected data, and the analysis scripts\footnote{[anonymized]}.

\indent\textbf{(1) Hand Tracking.} We used an HTC Vive Tracker, attached to the participant's wrist, to track their right hand at 90Hz. Therefore, we ran a simple Unity3D scene and streamed the data via the labstreaminglatyer (LSL)\footnote{https://github.com/sccn/labstreaminglayer}.

\indent\textbf{(2) EEG Recording.} EEG data was recorded from 64 actively amplified Ag/AgCl electrodes in an actiCap Snap cap using BrainAmp DC amplifiers from BrainProducts. Electrodes were placed according to the extended international 10–20 system \cite{Chatrian1985-ys}. One electrode was placed under the left eye to provide additional information about eye movements (vEOG). After fitting the cap, all electrodes were filled with conductive gel to ensure proper conductivity and electrode impedance was brought below 10k$\Omega$ for all electrodes. EEG data was recorded with a sampling rate of 250 Hz. We used LSL to make the EEG data stream available in the network and synchronize the recordings of EEG data, motion capture and an experiment marker stream that marked sections of the study procedure.

\missingfigure{show participant with measurement system and description next to each element}

\indent\textbf{(3) Electrical Muscle Stimulation.} We actuated the index finger via electrical muscle stimulation (EMS), which was delivered via two electrodes attached to the participants' extensor digitorum muscle. We utilized the extensor digitorum since we found that we can robustly actuate it without inducing parasitical motion of neighboring muscles. This finger actuation was achieved via a medically-compliant battery powered muscle stimulator (prorelax TENS/EMS Super Duo Plus). The EMS system's output was controlled by flipping a solid state relay connected via an Arduino Uno to the experiment computer. The EMS was pre-calibrated per participant to ensure a pain-free stimulation and robust actuation.

\subsection{Task}
\missingfigure{add an infographic of the task progression with a time axis here, this should be covering two columns and be at the top of the page!}
Participants performed a simple reaching task: resting their hand on the table, participants were instructed to press a specified key on a notebook in front of them. For each trial, a key was randomly selected from one of two groups of letters: (1) 'a', 's', 'd' or (2) 'j', 'k', 'l'. The selected group was then changed with every trial. This was done to keep the task engaging while maintaining a comparable reaching movement between trials.

The interaction flow for one trial of our task, depicted in Figure~\ref{task_flow}, was as follows: (1) participants waited for a new letter to appear on screen; (2) then, they were instructed to wait for a brief moment (~2s), before (3) reaching out and pressing the key. In line with the literature on the origin of the RP generating process they were told "to avoid preplanning the movement, avoid any obvious rhythm, and to press when they felt the spontaneous urge to move". (4) Two \todo{how many, but do keep this constant in order for them to focus on the duration of their action} seconds after they pressed the key, they were presented with a tone and asked to judge whether their movement, from the start of their reach to the button press, lasted longer or shorter than the tone. (5) When they perceived the tone to be longer, they had to press the up key or vice versa the down key for perceiving that the tone was shorter than their movement. Following an inter-trial-interval the next trial started.

\subsubsection{Experiment Design and Procedure}
Participants completed three experimental blocks. The first block consisted of 60 trials and was used to obtain training data for the single-trial classification system outlined below. 

%The condition of the second and third block, see below, were alternated between participants.

In a second block, participants completed 80 trials of the same task and trial order as in the first block. Now, the brain-computer interface (BCI) detecting readiness potentials from real-time EEG data was continuously running in the background. After the target letter was shown to the participants and they were preparing to reach out and press the corresponding key, the BCI was live. Whenever the BCI detected EEG signals to reflect an interaction intent, EMS was activated causing the participants index finger to lift. 


Both experimental blocks two and three were followed by a mixed methods data collection of participants' perceived agency.

In summary, the experiment consisted of four phases: (1) a setup phase; (2) a first experimental block: the training phase; (3) a second experiment block: the test phase; and (4) a test battery of questionnaire and interview to inquire about their experienced agency.

\subsection{Training of Classifiers from Training Phase Data}
% Brain-Computer Interface to Detect EEG Readiness Potentials }
% Classifiying EEG Signals of the Intent to Interact

Before the test phase of our experiment, the data recorded during the training phase was processed in three consecutive steps: (1) we extracted movement onsets in each trial to construct three event classes: an \textit{idle} class representing resting background activity, a \textit{movement} class and a \textit{pre-movement} class, representing the intent to interact. Next, (2) a motion classifier was trained on \textit{movement} and \textit{idle} data segments and (3) an EEG classifier was trained on \textit{pre-movement} and \textit{idle} data segments using 10 features from the 20 most informative EEG channels. To classify single-trial motion and EEG data we used a regularized linear discriminant analysis (LDA) that was trained for each participant individually. The EEG classifier was then used in the test phase to detect RPs and trigger EMS. The motion classifier was used to make the interaction more robust. When it detected the participants hand to be moving, it deactivated the control loop from RP to EMS.

\missingfigure{some picture from the setup and prototype}

\indent\textbf{(1) Labelling Pre-movement, Movement and Idle Data Segments.}
In order to obtain training data segments for the \textit{pre-movement} class, a detection algorithm was applied on the hand velocity time series. From the training data, the raw hand motion data was filtered with a 6Hz low-pass filter and re-sampled to match the EEG sample rate using BeMoBIL Pipeline functions \textit{xdf2bids} and \textit{bids2set}\footnote{https://github.com/BeMoBIL/bemobil-pipeline}. Subsequently the first derivative was computed and velocity was extracted. Then, for each trial a two step thresholding process was applied to extract the exact time of movement onset: (1) the first time point when the velocity exceeded 70\% of the maximum velocity in the trial was selected, then (2) the signal was flipped at that time point and the time point where the flipped signal first fell below 10\% of the maximum in the remaining data was selected as the movement onset.

The three event classes were then defined as follows: \textit{pre-movement} from -1000 to 0 ms preceding the movement onsets, \textit{movement} from 0 to 1000 ms succeeding the movement onsets and \textit{idle}, a one second data segment that fell exactly between two succeeding trials, i.e. time-locked to the inter-trial-interval. Here, participants where looking at a fixation cross on the screen with their hand resting on the table.

\indent\textbf{(2) Training of Hand Motion Classifier.}
An LDA was trained using the mean velocity per trial and movement and idle time segment. Hence, 60 values for each class were subjected to train the motion classifier. Since in the idle time segments almost no movement occurred we added some artificial noise. To this end, the idle features were multiplied by a factor of 3, chosen by simple trial and error. In the real-time feedback, this rendered the classifier less susceptible to false positives, i.e. detecting a movement when there was none.

\indent\textbf{(3) Channel Selection \& Training of EEG Classifier.}
For the EEG classifier, we followed the approach outlined by Schultze-Kraft et al.\cite{} with slight modifications. First, the 20 channels with the strongest discrimination between \textit{pre-movement} and \textit{idle} segments were selected with the following procedure. For both \textit{pre-movement} and \textit{idle} the mean signal in the last 100 ms of each segment was subtracted from the mean signal in the first 100ms of the segment. The resulting value thus indicated how much the signal had changed in the course of the 1s segments. In line with the literature there should be a change over time in the \textit{pre-movement} segment but not in the \textit{idle} segment. Then, all channels were sorted (1) in descending order by the signal difference in the \textit{pre-movement} segments and (2) in ascending order by the signal difference in the \textit{idle} segments. The idea was, that an ideal channel shows a strong change in signal over the course of a \textit{pre-movement} segment but shows no difference during a \textit{idle} segment. The ranks were joined and the 20 best channels were selected. In any case, channels C3, C4 and Cz were kept for every participant as these are most frequently reported in studies on the RP.

To train the LDA EEG classifier, we extracted spatiotemporal features from the EEG; 10 windowed means were extracted for each segment and channel. These were baseline corrected by subtracting the mean of the last 100 ms preceding the segment. The features were concatenated across all 20 selected channels and 10 time windows to obtain a 200 dimensional spatiotemporal feature vector per \textit{pre-movement} and \textit{idle} segment.

\subsubsection{Real-time Feedback}

During the test phase, both classifiers were applied to the real-time data. With the motion data streaming at 90 Hz we subjected the current sample to the motion classifier. For the EEG data streaming at 250 Hz, the EEG data of the last 1000ms was buffered for the selected best channels and 10 windowed means per channel were computed. Identical to the training data, these features were baseline corrected by subtracting the mean from the last 100 ms preceding the last second, i.e. -1100 ms to -1000 ms, resulting in a 200 dimensional baseline corrected spatiotemporal feature vector. Hence, both classifiers yielded one output value at each sample point.

The EMS was triggered whenever the EEG classifier predicted the \textit{pre-movement} class with an above .7 probability. Additionally, EMS was only triggered when the hand was \textit{not} moving. In each trial, as soon as the hand was in motion once after trial start, the EEG-EMS control was deactivated for the rest of the trial. This guaranteed that participants experienced EMS only during movement preparation.

\subsection{Interview \& Questionnaire}

To assess if agency was preserved, three measures were recorded during and after the test phase.

\indent\textbf{(1) Intentional Binding.}
At the end of each trial participants were tasked to compare the duration of their arm movement to a tone and indicate whether the tone was longer or shorter. We set the initial duration at 500 ms for each participant. When participants indicated the tone to be longer or shorter by pressing either the up or down key we added, or respectively subtracted, 100 ms from the tone duration. This new duration was then used for the next trial and so on. With 60 trials in the training phase, we hypothesized participants would oscillate 50-50 around a target duration after about 2/3 of the training trials. The final duration after all training trials was then used again as the start duration during the test phase. In line with the literature on intentional binding, we hypothesized this duration to remain stable in the case that agency was preserved during the test phase with EMS. A deviation of the tone duration from the training phase was hypothesized to reflect a disruption in experienced agency.

\indent\textbf{(2) Questionnaire.}
In line with~\cite{Hornbaek} we prompted participants with three statements and a 7-point Likert scale centered at 0. The first one ``I felt like the device I was using was part of my body'' was in relation to body ownership. The following two statements ``It felt like I was in control of the movements during the task.'' and ``What is the degree of control you felt?'' prompted participants on their experienced agency.

\indent\textbf{(3) Interview.}
We interviewed user's about the experience with the EMS integration. We asked open how questions in a collaborative interview inquiring about the user's lived experience. After prompting user's to recall their experience we started the interview with: "how did it feel like?". If user's made reference themselves to the EMS device we followed up by inquiring: "why do you think it (the EMS trigger) happened?", "What do you think triggered it?". The goal was to learn about the source attribution user's made about the EMS trigger. Since we leveraged the RP, we hypothesized that user's would attribute, at least in part, the EMS trigger to themselves, i.e. their movement intention. Next, we pressed deeper into learning about the cooperative nature between user and system. We asked "Do you believe you did it alone?" and when they made reference to a specific situation where the EMS was triggered we followed up with "What did it feel like, did you had the feeling the system controlled you or did you cooperate?" and "What would have to be there for it to feel like a cooperation?". We closed the interview with two additional exploratory questions: "Did the system influence your performance? If yes, how and why?" as well as asking them about any application scenarios they have in mind. 

To not break the user's immersion by disturbing the flow of the interaction we decided against asking qualitative questions at the end of each and every trial.

\subsection{Statistical Analyses}
\todo{how where the results analyzed: sentiment analyses and statistics of IB data and questionnaire results}

% ERP with significance mask for Cz? Assessing significance of classifier

% IB Task: ttest of final ib time between with and without EMS -> # subjects input values
% did they increase the IB task in more than 50% of the trials after EMS?

% questionnaires: ttest against zero to report direction

% interviews: anecdotal summary and sentiment analyses with two groups splitting questionnaires at 0

% correlation with EEG classifier accuracy?

% evaluation of EEG classifier: can report false negatives, how often was it missing to predict a movement onset? hand was moving without EMS being triggered occurred how often? Not possible to evlaute false positives as no ground truth available

\subsection{Reproducing Results and Data Availability}
Data, experimental protocol, analyses code including scripts for a reproduction of the presented results are hosted at open science foundation (OSF)\footnote{[anonymized]}. BIDS formatted data is hosted on openneuro \cite{}.