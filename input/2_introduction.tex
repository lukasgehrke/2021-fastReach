\section{Introduction}
% Lukas note:
% I like to use SCQA (Situation, Complication, Question, Answer) 'system' to write the introduction, see one of the two links below
% https://analytic-storytelling.com/scqa-what-is-it-how-does-it-work-and-how-can-it-help-me/
% https://corporatefinanceinstitute.com/resources/careers/how-to-job-guides/scqa/
% https://karrierebibel.de/scqa-methode/

% 1. Describe the Current Situtation
% Functions as a starting point and a common basis. Therefore it primarily contains recognizable and agreed points.

% 2. What is the complication, challenge identified
% Spells the reason for acting now. It contains threats / opportunities and the hurdles that need to be overcome.

% 3. Question
% Asks the question how the hurdles of the Complication can be overcome. How can we prevent the threat or seize the opportunity? Also, what would be the benefits if the complication would be overcome?

% 4. (short) Answer Teaser
% Provides the answer on how to overcome the hurdles. Explains how this will help deflect the threats or seize the opportunities.
% keep this short

\subsection{Main contribution: Our proposed solution}
% 4. (long) Anwser with teaser image
% Provides the answer on how to overcome the hurdles. Explains how this will help deflect the threats or seize the opportunities.
% Now extend into a longer paragraph with a teaser image explaining the system in the following subsection

% for citations use use "~\cite{}" instead of " \cite{}" → no white space, copy bibtex items to cite into input/bibliography.bib
% example:
some text~\cite{Gehrke2019-og}


% This is what most people (90+%) will see from your paper!

% - [ ]  use hemingwayapp.com (+SEO if hardcore) to make figure captions super precise
% - [ ]  are all the axes labels readable, font size 10 or up
% - [ ]  check for missing information: give someone the figure with captions to someone neutral to check if everything is clear from the figure alone
% - [ ]  for digital version: what is alt-text of figure, i.e. when hovering with the mouse over the figure what text appears next to the cursor
% - [ ]  for accessibility: write figure caption for the blind
%     - [ ]  explain what the figure shows
%     - [ ]  refactor using hemingwayapp.com

\begin{teaserfigure}
  \includegraphics[width=\textwidth]{figures/task_design.png}
  \caption{(a) description of the experimental conditions, (b) their hypothesized effects on sense of agency from the perspective of a comparator model, is inspired by~\citet{Dewey2019-dg} (c) depicts the flow of movement generation in the participant and the system between conditions.}
  \Description{(a) description of the experimental conditions, (b) their hypothesized effects on sense of agency from the perspective of a comparator model, is inspired by~\citet{Dewey2019-dg} (c) depicts the flow of movement generation in the participant and the system between conditions.}
  \label{fig:teaser}
\end{teaserfigure}