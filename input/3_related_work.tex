\section{Related Work}
Our research draws inspiration from neuroscience, especially in assessing the SoA qualitatively and quantitatively, and from engineering work on BCIs as well as on physical user augmentation.

\subsection{Theories of Sense of Agency}
The most widely used theory on how the SoA arises is the \textit{comparator model}~\cite{Blakemore2002-dj, Frith2000-ch, Frith2006-sc}: When we intentionally perform an action, the brain generates sensory predictions about the action outcome. These predictions are constantly compared to the actual sensory data available through the sensory system during the execution of the action. These include continuous signals such as proprioceptive and visual monitoring of the ongoing movement as well as higher level predictions about the semantic outcome of the action~\cite{Clark2013-ah, Haggard2003-ff, Haggard2017-uv}. If no sensorimotor incongruency arises, a SoA manifests. 

In the simple case of pressing a key on a piano, the finger movement is constantly compared to the predicted proprioceptive feedback. Subsequently, the tone generated by the key press is evaluated against auditory predictions. On a semantic level, these predictions may be in reference to whether the tone loudness corresponds to the velocity of the key press or whether the tone is in-key or out-of-key~\cite{Pangratz2023-ew}. If these predictions -- based on the intended movement and its expected outcome -- explain the sensory data available, agency is experienced.
%, see figure~\ref{fig:task_design} INTENTION).

In human-computer interaction (HCI) research, these constructs are categorized using a different terminology. Often, the term \textit{pre-reflective} is used to describe `early', implicit, experience of agency, such as when matching proprioceptive predictions about finger movements. At higher levels of the cognitive hierarchy, \textit{reflective}, i.e. conscious, experience is used to refer to matching semantic predictions about action outcomes~\cite{Danry2022-xk, Cornelio2022-aq}. 

As opposed to intended actions using our own body, movement augmentation hardware allows moving a user's body without their intention. Today, there are three main technologies to physically augment users' actions: Through the use of mechanical actuators, i.e., exoskeletons, a user's body can be moved by applying forces to the extremities. Another possibility is to stimulate the brain directly, so the stimulation causes a motor response, for example by using transcranial magnetic stimulation (TMS). Lastly, electrical muscle stimulation (EMS) makes the user's extremities move by sending current into their muscle-activating nerves. Irrespective of the method applied, the user computes no predictions about the movement and its outcome in these scenarios. Thus, concerning the comparator model, none or a decreased SoA arises in the case of externally controlled actions, for example by brain stimulation~\cite{Haggard2002-sz} or when the body is moved by another person~\cite{Kuhn2013-ls}. With something or somebody else moving our body to perform, e.g., a key press on a piano, the lack of intention to play the piano -- and derived sensory and semantic predictions -- negatively impacts the experience of agency. 

\subsubsection{Measuring Sense of Agency}
Both explicit and implicit methods have been developed to evaluate the sense of agency. These methods provide the basis to investigate the effects of action augmentation technology on agency experience. Explicit methods directly query participants to report their subjective experiences using questionnaires. Items such as ``It felt like I was in control of the hand I was looking at''~\cite{Haggard2002-sz} or ``Indicate how much it felt like moving the joystick caused the object on the computer screen to move''~\cite{Ebert2010-lu}, query either the \textit{pre-reflective} action -- or the \textit{reflective} outcome evaluation~\cite{Moore2012-dk}. However, in most cases such questionnaires aim at a higher-level, reflective, judgment of agency. 

On the other hand, implicit methods are often used to query low-level pre-reflective sensory predictions that are not consciously perceived~\cite{Moore2016-ub, Limerick2014-un, Moore2012-ic}. Seminal work in neuroscience has described one effect of SoA as a bias in the perception of action \textit{outcome}: Intentional binding paradigms state that when a button press is followed by a -- delayed -- outcome, participants mentally compress the delay~\cite{Haggard2002-sz}. Critically, this temporal compression only occurs following movements that were intended. The action outcome is mentally \textit{bound} to the intention. To reduce uncertainty about the binding, the brain `explains away' the excess delta, compressing the action-outcome delay~\cite{Barlas2018-bq}. Supplementing this behavioral phenomenon, physiological evidence can prove useful to further understand the interplay between volitional action and the SoA.

\subsection{Controlling Actuated Haptic Experiences}

Experimental setups to investigate new `on-body' augmentation technologies that aim to preserve the user's SoA frequently use highly controlled `stimulus-response' paradigms. For example, scenarios where participants are instructed to tap on a touchscreen in response to a presented stimulus on the screen. Here, participants' behavior can be predicted with very high certainty to follow the presented stimulus, and estimating their reaction time is very accurate. In such controlled scenarios, the timing of an action augmentation device can be tuned to be near optimal. Hence, \textit{pre-empting} the user's motion can be designed to fall in line with their intention to move, thereby maintaining SoA. Previously, ~\citet{Kasahara2019-sk} used a reaction time task in which participants had to tap a target on screen as soon as it appeared and subsequently rate their SoA. They showed that in such a scenario, user's actions can be pre-empted and that a pre-emption of about 80 ms best preserves agency~\cite{Kasahara2019-sk, Kasahara2021-gy}. 

This is in line with evidence from cognitive neuroscience which indicates that from around 200ms before a voluntary movement, users are unable to ``veto'' their self-initiated movement~\cite{Schultze-Kraft2016-bx}. Hence, after this ``point of no return'' user's struggle to assign a source other than themselves to the action initiation. Here, the \textit{key} aspect for SoA in action augmentation becomes apparent: External influences on the user's body need to be in line with the user's intention to act. Cruciall then, a key challenge remaining is to design systems that maintain agency when user's actions are unpredictable and where the experimenter does \textit{not} have executive control over the environment. In other words, how can a closed-loop system to deliver a \textit{natural} agency experience for users' augmented actions be designed?


\subsubsection{Using Brain Signals Reflecting the Intent to (Inter-)act for Action Augmentation}
One possible design solution is to leverage physiological signals for action augmentation. Of the possible physiological signals that can be leveraged, the EEG is very well suited because of its high temporal resolution and the non-invasive recording close to the motor command generating structures in the human brain. 

The RP, or \textit{lateralized readiness potential}, is an amplitude fluctuation in the ongoing EEG activity that has frequently been observed preceding voluntary action~\cite{Deecke1969-bl, Libet1983-qu}. The RP is reliably observed at electrodes placed over the sensorimotor cortex contralateral to the acting hand. In the extended 10-20 system for EEG electrode placement~\cite{Jasper1983-uw}, these are electrodes C3 located over the sensorimotor cortex of the left hemisphere, and C4 vice versa. However, activity observed at electrode Cz is reported most frequently as it reflects neural activity originating from the sensorimotor cortex without lateralization bias. Since the RPs' measurable onset precedes the time of participants' self-reported conscious movement intention, it has drawn much interest with respect to the debate on free will, see~\cite{Schurger2021-vp} for a recent neuroscientific perspective. However, evidence abounds for its role in action preparation. An RP is typically comprised of two stages: an early slow stage that begins up to two seconds before the actual movement and a late steep stage that starts about 400 milliseconds before movement. The first stage manifests in the pre-supplementary motor area and transfers to the premotor cortex shortly after. The second stage manifests contra-laterally in the primary motor cortex~\cite{Shibasaki2006-mt}. 

A recent study has shown that the RP is ingrained in the subconscious mechanisms preceding movements that people cannot explicitly suppress~\cite{Schultze-Kraft2021-cu}. In their study,~\citet{Schultze-Kraft2021-cu} asked participants to find a way to perform voluntary movements while keeping accompanying RP amplitudes as small as possible. After each trial they informed participants about the strength of the RP in the current trial, so participants had a feedback metric to optimize for. They found participants unable to suppress their RP. This inability to suppress the RP renders it a reliable feature for classification. For example, the RP can be detected in real-time using a brain-computer interface (BCI).~\citet{Schultze-Kraft2016-bx} demonstrated a prototype that detects RPs in participants ongoing EEG data and adapts an interface accordingly. In their study, participants were instructed to veto their self-initiated movement whenever a red dot occurred on the screen. The red dot's appearance was controlled by the BCI. Whenever an RP was detected, the red dot appeared. The authors found that participants were able to veto their self-initiated movement if the red dot appeared no later than 200ms preceding their movement onset. After that, participants were unable to ``overwrite'' their motor command and acted regardless of the red dot's appearance on screen.

Taken together, these findings demonstrated that the RP is a reliable signal preceding voluntary movement initiation and hence, is a well-suited candidate to base (neuro--)adaptive systems on.
