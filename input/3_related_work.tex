\section{Related Work}

% In this paper, we present a physical action augmentation prototype and a user study to investigate the experience of agency during the use of the prototype. Hence, our research builds on previous qualitative and quantitative research on the sense of agency.
% on actuated haptic systems, brain-computer interfacing as well as
% moving the finger of a user when the user intends to do so, but before the user directs their muscles to move.

The sense of agency, or this being in the "driving seat when it comes to our own actions"~\cite{Moore2016-ub} can be subdivided into the \textit{feeling} of agency, a low level pre-reflective sensory process, and a more higher level reflective cognitive process, the \textit{judgement} of agency~\cite{Moore2016-ub, Danry2022-xk}. 

\subsection{Measuring agency experience}
% explicit measures - reflective (short paragraph)
% - questionnaires
% - phenomenological interview



% implicit measures - pre-reflective (longer paragraph)
% - intentional binding: explain what this is and give a bit more detail about the underlying mechanisms and ideas about whats going on in the brain here

\textit{Intentional binding} is the phenomenon of subjectively compressing the time interval between a voluntary action and the sensory consequences of that action~\cite{Moore2012-ic}.

% what does that have to do with agency?
\subsection{Sense of agency and actuated haptic systems}

% close this paragraph with the problem that of the control mechanism of the 
