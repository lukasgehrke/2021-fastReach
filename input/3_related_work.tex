\section{Related Work}

We present a bodily integration experience moving the finger of a user after the user intends to do so, but before the user directs its muscles to move.

\subsection{The Disruption in Sense of Agency in Augmented Humans}

\subsection{Assessing the Sense of Agency when Integrating with Computers}
% how can we measure our outcome measure sense of agency?
% - questionnaires
% - intentional binding
% - phenomenological interview


\subsection{Brain-Computer Interfaces for Intent Prediction}
% Single-trial Classification of the Intent to Interact
Our idea to overcome the disruption in agency is to use EEG signals reflecting (inter)action intent. One such signal is the Readiness Potential (RP), or \textit{lateralized readiness potential}. It is an amplitude fluctuation that has been frequently observed preceding voluntary action~\cite{Deecke1969-bl, Libet1983-qu}. It is reliably observed in electrodes placed over the motor cortex contralateral to the acting hand. Since its measurable onset precedes the time of participants self-reported conscious movement intention, it has drawn much interest with respect to the debate on free will, see~\cite{Schurger2021-vp} for a recent neuroscientific perspective. However, evidence abounds for its role in action preparation. An RP is typically comprised of two stages: an early slow stage that begins up to two seconds before the actual movement and a late steep stage that starts about 400 milliseconds before movement. The first stage manifests in the pre-supplementary motor area and transfers to the premotor cortex shortly after. The second stage manifests contra-laterally in the primary motor cortex~\cite{Shibasaki2006-mt}. A recent study has shown that the RP is heavily ingrained in the subconscious mechanisms preceding movements that people cannot explicitly suppress~\cite{Schultze-Kraft2021-cu}. In their study,~\cite{Schultze-Kraft2021-cu} asked participants to find a way to perform a voluntary movement so that the accompanying RP amplitude was as small as possible. After each trial they informed participants about the strength of the RP in the current trial, so participants had a feedback metric to optimize for. They found participants \textit{not} to be able to suppress their RP. This inability to suppress the RP renders it a reliable feature for classification. While its role in movement preparation remains somewhat controversial, it is established as a signal occurring during movement planning and motivates a usage as a reliable predictor of movement.
