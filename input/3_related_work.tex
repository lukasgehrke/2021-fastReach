\section{Related Work}

% In this paper, we present a physical action augmentation prototype and a user study to investigate the experience of agency during the use of the prototype. Hence, our research builds on previous qualitative and quantitative research on the sense of agency.
% on actuated haptic systems, brain-computer interfacing as well as
% moving the finger of a user when the user intends to do so, but before the user directs their muscles to move.

Our research builds on previous engineering work in brain-computer interfacing and on theories of the sense of agency as discussed in cognitive science and neuroscience.

\subsection{The sense of agency in human-computer interaction}
% LT to write first version
\begin{comment}

%%% theory on sense of agency 
Agency is largely explained with a comparator model, describing internal comutational mechanisms of human action control 
-- comparator model (Blakemore 2002, Frith 2000, Frith 2005)
-- integrate predictive coding??

\subsection{neural basis of sense of agency}

\end{comment}

The sense of agency, or this being in the "driving seat when it comes to our own actions"~\cite{Moore2016-ub} can be subdivided into the \textit{feeling} of agency, a low-level pre-reflective sensory process, and a more higher level reflective cognitive process, the \textit{judgement} of agency~\cite{Moore2016-ub, Danry2022-xk}.



\subsection{Measuring agency experience}
% LT to write first version
\begin{comment}


% explicit measures - reflective (short paragraph)
% - questionnaires

%% from bergström 22
- read 18, 37 from  Bergstöem 
- 14,14,19.31,39,41,48
11,20,33

Sense of control is mostly measured with simple items with ratings like "It felt like I was in control of the hand I was looking at” rated on a 7-point Likert scale, +3 indicating strong agreement and −3 indicating strong disagreement (Longo and Haggard) or  "Indicate how much it felt like moving the joystick caused the object on the computer screen to move” as a measure of explicit sense of agency, also rated on a 7-point scale" (11) 
item asking participants to indicate the degree of control they have felt over the changes on the screen [1]. We used this item also as an agency measure (1) 

Individuals’ sense of agency can be estimated with explicit self-reported measures using Likert’s scales or percentages ( van der Wel et al., 2012; Chambon et al., 2013),“ 



% - phenomenological interview

% implicit measures - pre-reflective (longer paragraph)
-subjective measures address a high level-SoA  based on subjective judgments of feeling of control (Barlas and Kopp)
- proxy for the low-level SoA as they do not require conscious reflection on one’s SoA (Synofzik et al., 2008a; Desantis et al., 2011; Moore and Obhi, 2012; Barlas and Obhi, 2014).“ ([Barlas und Kopp, 2018, p. 2]
- one such measure is intentional binding

% - intentional binding: explain what this is and give a bit more detail about the underlying mechanisms and ideas about whats going on in the brain here
- „refers to the perceived temporal attraction between voluntary actions and their outcomes (Haggard et al., 2002). More clearly, the temporal interval between actions and outcomes is perceived as shorter when these outcomes are produced by voluntary actions compared to when they follow, for instance, involuntary movements or external causes (Haggard et al., 2002).“ ([Barlas und Kopp, 2018, p. 2]
- While details about the relationship between measures of intentional binding and SoA are far from clear, they are extensivly used 


- SoA is influeced by different processes, 
    - one of the is the source of action selection (self selected vs. externally instructed) (Chambon et al. 2014, Harrad 2017)
    - another: motor control processes (read Sidarus 2017a)
    - 


-


- Read More 2009 (interval estimations were shorter (i.e., stronger binding) in voluntary than involuntary movements) 
- add figuure action binding and effect binding (e.g.  in WEN and Imamizu)
- 

\textit{Intentional binding} is the phenomenon of subjectively compressing the time interval between a voluntary action and the sensory consequences of that action~\cite{Moore2012-ic}.
When a voluntary action is causally linked with a sensory outcome, the action and its consequent effect are perceived as being closer together in time. This effect is called intentional binding.“ (from Jo 2014)






% what does that have to do with agency?

- intentional binding effects are are sometimes descrepant from explicit judgment of agency (see 30-33 in Wen adn Imamizu) (Saito 2015; Majchrowitz 2018; Ebert
- Intentional binding is e.g. influenced by causal relations between events, attention, arousal (see Wen and Imamizu)

- EEG patterns related to readiness potential and intentional  binding sometimes show disceprant patterns ( see WEN and Imamizu 38 und 39) (wittmann 2014, Goldberg 2017)


- alternatives to intentional binding - sensory attenuation / visual attention





% RESEARCH GAP


- reach "we-mode" in human-machine joint control ---> decode human intention (Zander 2011, Felke 2019, Shiskin 2016)
- maintain sense of agency in augmented interactions 
- doing this by leveraging action intention in brain


----
involuntary movements produce less binding than do voluntary actions, or even reverse the effect entirely (Yoshi und Harrard 2013, Moore, Wegner und Harrard 2009)
it has been shown, that mere peripheral body movements, elicted by TMS simulations, produce a perceptual repulsion oppisite to intentional binding in truly operant intentional actions (Haggard, Clark 2002)

There is an ongoing debate, whether the temporal attraction is specific to intentional movements, or is more generally related to the perception between action and outcome

- 


\subsection{Sense of agency and actuated haptic systems}


\ subsection{Movement Intention}
idea: when we now compare completely voluntary movements with movements triggerd by the brain signal of our intention to move - what happens?

% close this paragraph with the problem that of the control mechanism of the actuator hardware and the disruotion in agency

% PosssessedHand, Gilbert-I miss being me, 

\end{comment}