\section{Related Work}

\subsection{Disrupted Sense of Agency in Human Augmentation}

This defines the concept of \textbf{natural augmentation} which pairs the idea of wearable designs with reduced or non-existent physical and cognitive burden when using the interface. This notion was illustrated by~\citet{Eden2021-og} which defined natural/true degree of freedom augmentation for movement as a usable additional interface that would be controlled by the user without sacrificing other movements. For example, the control of a robotic arm via a remote control is sacrificing the degree of freedom of our hands. However, a robotic arm that is controlled as if it was an additional identical separate limb would not impair our initial range of potential movements, thus it would be a true augmentation. We could extend this definition saying that any true augmentation would be an enhancement in the way to interact with the environment without impairing any other physical or mental function.

\subsection{Single-trial Classification of the Intent to Interact}

Our idea to overcome the disruption in agency is to use EEG signal reflecting (inter)action intent.

One possible usage scenario of BCIs can be realized by using the Readiness Potential (RP) to de- tect motion intent and trigger actions accordingly. The is a type of potential that has been known for decades (Kornhuber & Deecke, 1965), with research continuously providing evidence of its role in volitional movement preparation (Libet et al., 1993). While it is similar to an event-related potential (ERP), it does not get generated after a stimulus but before the conscious movement starts (Smulders et al., 2012). RP comprises two stages: an early slow stage that begins up to two seconds before the actual movement and a late steep stage that starts about 400 milliseconds before movement. The first stage manifests in the pre-supplementary motor area and transfers to the premotor cortex shortly af- ter. The second stage manifests contralaterally in the primary motor cortex (Shibasaki & Hallett, 2006).
A recent study has shown that the RP is heavily ingrained in the subconscious mechanisms preced- ing movements that people cannot explicitly suppress (Schultze-Kraft et al., 2021). The inability to suppress the RP indicates that it can always be present in the brain waves before motion onset, and it can serve as a reliable predictor of movement.
Having such a predictor for movement can open up new opportunities for the design of BCIs. Shih et al. (2012) defines BCIs as computer-based systems that acquire brain signals, analyze them, and translate them into commands relayed to an output device to carry out the desired action. The purpose of these systems is to detect and quantify the features of brain signals that indicate the user’s intention and translate these features in real-time into device commands that accomplish the user’s intent (Shih et al., 2012). Therefore, using the RP as a feature for prediction motion intent seems a good design for a BCI. Bai et al. (2011) have been the first to successfully set up such an interface and confirmed its feasibility, and Gerrits (2017) has replicated the setup in a follow-up experiment.

\subsection{Levels of Consciousness in HCI}

Along with this idea,~\citet{Jain2020-og} highlighted that numerous types of HCIs for human augmentation remain unexplored. Following their thoughts, they proposed a finer classification of the traditionally defined levels of consciousness (i.e. unconscious, conscious and meta-conscious) by splitting the unconscious into subliminal and pre-conscious and the meta-conscious into meta-cognitive and meta-somatic (Figure \ref{fig:consciousness}). The subliminal level corresponds to what is outside the range of perception of the user and thus a subliminal stimulus do not increase the cognitive load (e.g effect of a vitamin pill intake). A pre-conscious stimulus is below the awareness of the user but it becomes perceivable if the user change its focus toward it (e.g. the side of the road while driving). The conscious level occupies what the user is currently focusing on and thus most of the cognitive load comes from it. The meta-cognitive process is the awareness of the current conscious state (e.g. reflecting on the speed when walking). Finally, the meta-somatic pathway corresponds to the automatic response of the body to external stimulus, thus without intervention of the user (e.g. chills in response to a cold breeze).
This extended classification puts the spot on the new aspects on which a HCI can be designed and how they could use these pathways as new alternatives to contribute to natural human augmentation. Indeed, while numerous HCIs depend on the conscious state requiring the full user's awareness, the achievement of natural enhanced user experience may rely on the exploration of the interaction with other levels of consciousness.