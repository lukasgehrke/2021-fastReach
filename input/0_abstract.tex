
% Besides the title, this is what most people (90+%) will read from your paper!

% - [ ] Therefore, improve SEO (Search engine optimization): copy the abstract text to hemingwayapp.com to improve information density, keyword frequency and readability! Go through each sentence to shorten it and remove unnecessary words.
To maintain a sense of agency (SoA), physical motor augmentation must align with users' expectations. In experiments, this is often achieved using highly structured and controlled stimulus-response paradigms. Here, the users' motor behavior strictly follows a presented stimulus. To allow user augmentation in unstructured environments, we developed a brain-computer interface that controls the user's movement through muscle stimulation at the time of their intention to move without external stimuli. In a pilot study, we investigated whether our INTENTION system is capable of maintaining SoA when participants work with the system to tap a touchscreen. We conducted an intentional binding task and interviewed users. We found (a) intentional binding to occur when participants worked with the system but not when they were passively moved, and (b) that the level of control working with the system was perceived to be higher than when being passively moved, as indicated by subjective responses. Our closed-loop system enables novel predictive interfaces that leverage the users’ body directly yet still feel natural as they align with the users’ intention.

% To maintain a sense of agency (SoA), physical motor augmentation must align with users' expectations. In experiments, this is often achieved using controlled stimulus-response paradigms. Here, the users' motor behavior strictly follows a presented stimulus.



% \todo{What is the specific problem addressed?}
% In order to maintain a sense of agency (SoA), physical motor augmentation must be aligned with users' expectations. In experiments, this is often achieved using controlled stimulus-response paradigms where the users' motor behavior strictly follows a previously presented stimulus.

% \todo{What have you done?}
% To overcome this need for control, we developed a brain-computer interface (BCI) that establishes a fast communication channel between a user’s brain signals and a physical end effector. By classifying brain activity, readiness potentials (RP), the BCI controls the user’s muscles at the time of their intent to interact. 

% In a small user study, we investigated whether our system is capable to maintain SoA when participants work with the system to tap a touchscreen. We conducted an intentional binding task and interviewed users.

% \todo{What did you find out?}
% \todo{What are the implications on a larger scale?}


% While hardware advancements hold promise for overcoming human limitations and enhancing skills, users often report dissociative experiences that disrupt their SoA. The study introduces a novel approach using brain-computer interfaces (BCI) to control electrical muscle stimulation (EMS) precisely at moments when users intend to interact. Results show that intentional binding, a marker of SoA, is maintained when users actively engage with the system. However, the perceived level of control is lower compared to voluntary interaction without EMS when the stimulation does not align with user intentions. This research presents a cost-effective prototype and emphasizes the importance of aligning technology with users' intentions for successful integration, highlighting the evolving relationship between humans and AI technologies in future applications.