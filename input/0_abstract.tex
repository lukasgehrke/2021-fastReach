To maintain a sense of agency (SoA), physical motor augmentation must align with the users' expectations. In experimental setups, this is often achieved using stimulus-response paradigms where the user behavior is predictable. Moving towards user augmentation in the everyday world, we developed a brain-computer interface (BCI) based augmentation using electrical muscle stimulation (EMS). By classifying readiness potentials from the user's electroencephalogram (EEG), our system controlled the user's movements at the time of their intent to interact. The system was able to discriminate pre-movement from idle EEG segments with an F1 score of 0.7. In a pilot study, we observed intentional binding, a compression of time between action and outcome, and a higher level of control when participants worked with the system instead of being passively moved. This is a step towards predictive interfaces that leverage the users’ bodies yet still feel natural as they align with the users’ intentions.

