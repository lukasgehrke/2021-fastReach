% To maintain a sense of agency (SoA), physical motor augmentation must align with the users' expectations. In experimental setups, this is often achieved using stimulus-response paradigms where the user behavior is predictable. Moving towards user augmentation in the everyday world where user behavior is less predictable, we developed a brain-computer interface (BCI) based augmentation using electrical muscle stimulation (EMS). By classifying readiness potentials from the user's electroencephalogram (EEG), our system controlled the user's movements at the time of their intent to interact. The system was able to discriminate pre-movement from idle EEG segments with an F1 score of 0.7. In a pilot study, we observed intentional binding, a compression of time between action and outcome, and a higher level of control when participants worked with the system instead of being passively moved. This is a step towards predictive interfaces that leverage the users’ bodies yet still feel natural as they align with the users’ intentions.

To maintain a user's sense of agency (SoA) when working with a physical motor augmentation device, the actuation must align with the user's intentions. In experiments, this is often achieved using stimulus-response paradigms where the motor augmentation can be optimally timed. However, in the everyday world users primarily act at their own volition. We designed a closed-loop system for motor augmentation using an EEG-based brain-computer interface (BCI) to cue users' volitional finger tapping. Relying on the readiness potentials, the system autonomously cued the finger movement at the time of the intent to interact via electrical muscle stimulation (EMS). The prototype discriminated pre-movement from idle EEG segments with an average F1 score of 0.7. However, we found only weak evidence for a maintained SoA. Still, participants reported a higher level of control when working with the system instead of being passively moved.

% To maintain a user's sense of agency (SoA) when working with a physical motor augmentation device, the actuation must align with the user's intentions. In experiments, this is often achieved using stimulus-response paradigms where the user's behavior systematically follows a stimulus. Here, motor augmentation can be delivered at optimal timings. However, in the everyday world, user's primarily act and move on their own volition. We designed a closed-loop system for motor augmentation using an EEG-based brain-computer interface (BCI) to cue user's finger tapping. Relying on the user's readiness potentials, the system autonomously cues the user's finger movement at the time of their intent to interact via electrical muscle stimulation (EMS). Our prototype discriminated pre-movement from idle EEG segments with an average F1 score of 0.7. In a pilot study, we found weak evidence for a maintained SoA. Participants reported a higher level of control when working with the system instead of being passively moved. Additionally, we assessed intentional binding but did not find a systematic temporal compression between action and outcome. Our prototype is a further step towards closed-loop stimulation that directly acts on the users’ bodies while maintaining their SoA.