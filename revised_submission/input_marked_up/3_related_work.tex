\section{Related Work}
Our research draws inspiration from neuroscience, especially in assessing the SoA qualitatively and quantitatively, and from engineering work on BCIs as well as on physical user augmentation.

\subsection{Theories of Sense of Agency}
The most widely used theory on how the SoA arises is the \textit{comparator model}~\cite{Blakemore2002-dj, Frith2000-ch, Frith2006-sc}: when we intentionally perform an action, the brain generates sensory predictions about the action outcome. These predictions are constantly compared to the actual sensory data available through the sensory system during the execution of the action. These include continuous signals such as proprioceptive and visual monitoring of the ongoing movement as well as higher level predictions about the semantic outcome of the action~\cite{Clark2013-ah, Haggard2003-ff, Haggard2017-uv}. If no sensorimotor incongruency arises, a SoA manifests. 

In the simple case of pressing a key on a piano, the finger movement is constantly compared to the predicted proprioceptive feedback. Subsequently, the tone generated by the key press is evaluated against auditory predictions. On a semantic level, these predictions may be in reference to whether the tone loudness corresponds to the velocity of the key press or whether the tone is in-key or out-of-key~\cite{Pangratz2023-ew}. If these predictions -- based on the intended movement and its expected outcome -- explain the sensory data available, agency is experienced.
%, see figure~\ref{fig:task_design} INTENTION).

In human-computer interaction (HCI) research, these constructs are categorized using a different terminology. Often, the term \textit{pre-reflective} is used to describe `early', implicit, experience of agency, such as when matching proprioceptive predictions about finger movements. At higher levels of the cognitive hierarchy, \textit{reflective}, i.e. conscious, experience is used to refer to matching semantic predictions about action outcomes~\cite{Danry2022-xk, Cornelio2022-aq}. 

As opposed to intended actions using our own body, movement augmentation hardware allows moving a user's body without their intention. Today, there are three main technologies to physically augment users' actions: Through the use of mechanical actuators, i.e. exoskeletons, a user's body can be moved by applying forces to the extremities. Another possibility is to stimulate the brain directly, so the stimulation causes a motor response, for example by using transcranial magnetic stimulation (TMS). Lastly, electrical muscle stimulation (EMS) makes the user's extremities move by sending current into their muscle-activating nerves. In these scenarios, the user computes no predictions about the movement and its outcome. Thus, concerning the comparator model, none or a decreased SoA arises in the case of externally controlled actions, for example by brain stimulation~\cite{Haggard2002-sz} or when the body is moved by another person~\cite{Kuhn2013-ls}. With something or somebody else moving our body to perform, e.g., a key press on a piano, the lack of intention to play the piano -- and derived sensory and semantic predictions -- impacts the experience of agency. ~\textcolor{red}{Such involuntary movements may even cause adverse effects: In case a body movement is elicited by TMS simulation, a resistance to the movement may manifest~\cite{Haggard2002-sz}.}
%, see figure~\ref{fig:task_design} EXTERNAL CONTROL)}

\subsubsection{Measuring Sense of Agency}
Both explicit and implicit methods have been developed to evaluate the sense of agency. These methods provide the basis to investigate the effects of action augmentation technology on agency experience. Explicit methods directly query participants to report their subjective experiences using questionnaires. Items such as ``It felt like I was in control of the hand I was looking at''~\cite{Haggard2002-sz} or ``Indicate how much it felt like moving the joystick caused the object on the computer screen to move''~\cite{Ebert2010-lu}, query either the \textit{pre-reflective} action -- or the \textit{reflective} outcome evaluation~\cite{Moore2012-dk}. However, in most cases such questionnaires aim at a higher-level, reflective, judgment of agency. 

On the other hand, implicit methods are often used to query low-level pre-reflective sensory predictions that are not consciously perceived~\cite{Moore2016-ub, Limerick2014-un, Moore2012-ic}. Seminal work in neuroscience has described one effect of SoA as a bias in the perception of action \textit{outcome}: Intentional binding paradigms state that when a button press is followed by a -- delayed -- outcome, participants mentally compress the delay~\cite{Haggard2002-sz}. Critically, this temporal compression only occurs following movements that were intended. The action outcome is mentally \textit{bound} to the intention. To reduce uncertainty about the binding, the brain `explains away' the excess delta, compressing the action-outcome delay~\cite{Barlas2018-bq}. Supplementing this behavioral phenomenon, physiological evidence can prove useful to further understand the interplay between volitional action and the SoA.

\subsection{\textcolor{green}{Controlling Actuated Haptic Experiences}}

\textcolor{green}{Experimental setups to investigate new `on-body' augmentation technologies that aim to preserve the user's SoA frequently use highly controlled `stimulus-response' paradigms. For example, scenarios where participants are instructed to tap on a touchscreen in response to a presented stimulus on the screen. Here, participants' behavior can be predicted with very high certainty to follow the presented stimulus, and estimating their reaction time is very accurate. In such controlled scenarios, the timing of an action augmentation device can be tuned to be near optimal. Hence, \textit{pre-empting} the user's motion can be designed to fall in line with their intention to move, thereby maintaining SoA. Previous work on such an augmentation scenario has shown that an actuation preceding a movement by about 80 ms preserves agency~\cite{Kasahara2019-sk, Kasahara2021-gy}.}

\textcolor{green}{Evidence from cognitive neuroscience indicates that from around 200ms before a voluntary movement, users are unable to ``veto'' their self-initiated movement~\cite{Schultze-Kraft2016-bx}. Here, the \textit{key} aspect for SoA in action augmentation becomes apparent: External influences on the user's body need to be in line with the user's intention. However, a key challenge remaining is to design systems that maintain agency during unpredictable actions of the user and \textit{without} control over the environment. In other words, how can a closed-loop system to deliver a \textit{natural} agency experience for users' augmented actions be designed?}

\textcolor{red}{\subsection{Controlling Actuated Haptic Experiences using Brain Signals Reflecting the Intent to (Inter-)act}}

\subsubsection{\textcolor{green}{Using Brain Signals Reflecting the Intent to (Inter-)act for Action Augmentation}}
\textcolor{green}{One possible design solution is to leverage physiological signals for action augmentation}. Of the possible physiological signals that can be leveraged\textcolor{red}{used to further investigate the experience of agency}, the EEG is very well suited because of its high temporal resolution and the non-invasive recording close to the motor command generating structures in the human brain. The RP, or \textit{lateralized readiness potential}, is an amplitude fluctuation in the ongoing EEG activity that has frequently been observed preceding voluntary action~\cite{Deecke1969-bl, Libet1983-qu}. The RP is reliably observed at electrodes placed over the \textcolor{green}{sensori}motor cortex contralateral to the acting hand.~\textcolor{green}{In the 10-20 system for EEG electrode placement~\cite{Jasper1983-uw}, these are electrode C3 located over the sensorimotor cortex of the left hemisphere, and C4 vice versa. However, activity observed at electrode Cz is reported most frequently as it reflects neural activity originating from the sensorimotor cortex without lateralization bias.} Since the RPs' measurable onset precedes the time of participants' self-reported conscious movement intention, it has drawn much interest with respect to the debate on free will, see~\cite{Schurger2021-vp} for a recent neuroscientific perspective. However, evidence abounds for its role in action preparation. An RP is typically comprised of two stages: an early slow stage that begins up to two seconds before the actual movement and a late steep stage that starts about 400 milliseconds before movement. The first stage manifests in the pre-supplementary motor area and transfers to the premotor cortex shortly after. The second stage manifests contra-laterally in the primary motor cortex~\cite{Shibasaki2006-mt}. 

A recent study has shown that the RP is ingrained in the subconscious mechanisms preceding movements that people cannot explicitly suppress~\cite{Schultze-Kraft2021-cu}. In their study,~\citet{Schultze-Kraft2021-cu} asked participants to find a way to perform voluntary movements while keeping accompanying RP amplitudes as small as possible. After each trial they informed participants about the strength of the RP in the current trial, so participants had a feedback metric to optimize for. They found participants unable to suppress their RP. This inability to suppress the RP renders it a reliable feature for classification. For example, the RP can be detected in real-time using a brain-computer interface (BCI).~\citet{Schultze-Kraft2016-bx} demonstrated a prototype that detects RPs in participants ongoing EEG data and adapts an interface accordingly. Participants were instructed to veto their self-initiated movement whenever a red dot occurred on the screen. The red dot's appearance was controlled by the BCI. Whenever an RP was detected, the red dot appeared. The authors found that participants were able to veto their self-initiated movement if the red dot appeared no later than 200ms preceding their movement onset. After that, participants were unable to ``overwrite'' their motor command and acted regardless of the red dot's appearance on screen.

\textcolor{red}{In the present study, we leveraged the RP in a similar way, i.e. to discriminate between two \textit{action} states as the basis to trigger an action augmentation device or not. The technology we used with our system, electrical muscle stimulation (EMS), sends currents to the user's forearm leading to a contraction of the flexor muscle moving the ring finger as an augmented movement.}

%%% resources
\begin{comment}

(see Carruthers 2012 for a summary of arguments against this model)

There is an ongoing debate, whether the temporal attraction is specific to intentional movements, or is more generally related to the perception between action and outcome (Buhner 2012)
--> they suggest intentional action is not necessary for temporal binding, but that the binding results from a causal relation linking actions with consequences -- more general causal binding -- do we need to go there?

One idea: - reach "we-mode" in human-machine joint control ---> decode human intention (Zander 2011, Felke 2019, Shiskin 2016)
- maintain sense of agency in augmented interactions 
- doing this by leveraging action intention in brain

Experiencing control is an important factor in human-computer interaction (see Shneiderman and Nielsen). 

%%%%%%%%%% theory on sense of agency  / neural basis of sense of agency

% explicit measures - reflective (short paragraph)
% - questionnaires

%% from bergström 22
- read 18, 37 from  Bergstöem 
- 14,14,19.31,39,41,48
11,20,33

Sense of control is mostly measured with simple items with ratings like "It felt like I was in control of the hand I was looking at” rated on a 7-point Likert scale, +3 indicating strong agreement and −3 indicating strong disagreement (Longo and Haggard) or  "Indicate how much it felt like moving the joystick caused the object on the computer screen to move” as a measure of explicit sense of agency, also rated on a 7-point scale" (11) 
item asking participants to indicate the degree of control they have felt over the changes on the screen [1]. We used this item also as an agency measure (1) 

Researchers use implicit and explicit methods. Explicit methods involve directly asking participants to report their subjective experiences using items rated on Likert scales or percentages. These items, such as "It felt like I was in control of the hand I was looking at" (Longo & Haggard) and "Indicate how much it felt like moving the joystick caused the object on the computer screen to move" (Ebert & Wegner), may focus on either the action element or the outcome element of the sense of agency (see Moore).

Items such as ""It felt like I was in control of the hand I was looking at” (longo Haggard) and "Indicate how much it felt like moving the joystick caused the object on the computer screen to move” (Ebert und Wegner). Notably, such items differ between focusing either on the action element or the outcome element (See MOore)

% - phenomenological interview

% implicit measures - pre-reflective (longer paragraph)
-subjective measures address a high level-SoA  based on subjective judgments of feeling of control (Barlas and Kopp)
- proxy for the low-level SoA as they do not require conscious reflection on one’s SoA (Synofzik et al., 2008a; Desantis et al., 2011; Moore and Obhi, 2012; Barlas and Obhi, 2014).“ ([Barlas und Kopp, 2018, p. 2]
- one such measure is intentional binding

% - intentional binding: explain what this is and give a bit more detail about the underlying mechanisms and ideas about whats going on in the brain here
- „refers to the perceived temporal attraction between voluntary actions and their outcomes (Haggard et al., 2002). More clearly, the temporal interval between actions and outcomes is perceived as shorter when these outcomes are produced by voluntary actions compared to when they follow, for instance, involuntary movements or external causes (Haggard et al., 2002).“ ([Barlas und Kopp, 2018, p. 2]
- While details about the relationship between measures of intentional binding and SoA are far from clear, they are extensivly used 
- alternatives to intentional binding - sensory attenuation / visual attention
- involuntary movements produce less binding than do voluntary actions, or even reverse the effect entirely (Yoshi und Harrard 2013, Moore, Wegner und Harrard 2009)
it has been shown, that mere peripheral body movements, elicited by TMS simulations, produce a perceptual repulsion opposite to intentional binding in truly operant intentional actions (Haggard, Clark 2002)

\textit{Intentional binding} is the phenomenon of subjectively compressing the time interval between a voluntary action and the sensory consequences of that action~\cite{Moore2012-ic}.
- comparisons between active and passive finger tab 
    Kühn 2013 + Engbert 2007 : experimenter presses the finger down in a passive condition
    
When a voluntary action is causally linked with a sensory outcome, the action and its consequent effect are perceived as being closer together in time. This effect is called intentional binding.“ (from Jo 2014)

- Read More 2009 (interval estimations were shorter (i.e., stronger binding) in voluntary than involuntary movements) 
- add figuure action binding and effect binding (e.g.  in WEN and Imamizu)


!! new paper suggesting that temporal binding / intentional binding is not a measure of sense of agency !!(Gutzeit!)  


% connection between implicit and explicit measures:
- intentional binding effects are sometimes discrepant from explicit judgment of agency (see 30-33 in Wen and Imamizu) (Saito 2015; Majchrowitz 2018; Ebert
- Intentional binding is e.g. influenced by causal relations between events, attention, and arousal (see Wen and Imamizu)


% Interesting HCI papers:
% PosssessedHand, Gilbert-I miss being me, 


% Why is sense of agency important? Why would it be bad if we lost it in augmented humans?
% However, for augmentation hardware to be readily accepted, maintaining a sense of agency is important for several reasons. (find reasons)
% - motor learning ( check Wen, W. et al. Perception and control: individual difference in the sense of agency is associated with learnability in sensorimotor adaptation. Sci. Rep. 11 , 20542 (2021).
% - motivation (find sources)

%%%%%%%%%%%%%%%%%%%%%%
Other
- EEG patterns related to readiness potential and intentional  binding sometimes show disceprant patterns ( see WEN and Imamizu 38 und 39) (wittmann 2014, Goldberg 2017)


% This straightforward relationship is complicated by the introduction of technology. Think about, playing a computer game where a key press controls an avatar playing the piano. Here, we do not really press the piano key, the avatar's movement, and the tone are now generated by a computer. Despite the blurring lines between the performer and the cause of the tone, we still feel a sense of authorship when the comparison does not reveal incongruency. This lingering feeling of control is thought to be elicited by our intentions being reflected in the resulting action (sources needed). 

% alternative headings:
% Controlling Actuated Haptic experiences using brain signals reflecting the intent to interact
% Triggering EMS using EEG Signals

\end{comment}