This work presents a BCI-based augmentation that is able to stimulate users' muscles at their intent to interact. The system measured readiness potential (RP) from EEG signals to calculate action intention. The authors conducted a user study to measure intentional binding while interacting. The results show a higher level of control working with the system compared to when being passively moved (via EMS).

I find the goal of this work is well-motivated and would fit into CHI. All reviewers appraise the work for its ambitious goal (2AC), has a sensible approach, overall aim interesting (R1), and introduce an original system that can be an important step towards action augmentation using neural input (R2).

However, the paper is well-written and has good-quality research. However, there are some issues that should be addressed in the revision:

** 2AC: 
It is unclear if both EMS triggering and EMG measurements were performed in the AUGMENTED condition. If so, how that was done technically should be explained.

The authors did a comparison between three conditions. However, how those conditions are compared to other techniques (as a baseline) in previous studies should be discussed.

Both 2AC and R1 are concerned with a justification of the classification rate (10Hz) and the maximum interaction time reduction that this technique can achieve.

** R1: 
I agree with the R1's concern regarding the low number of participants (8 people). As the trials were repeatedly measured and analysed (independently?), would the means of those trials (i.e., time estimation and sense of control rating) should also be used?

The used timing in the EXTERNAL condition was not particularly extreme and randomly picked from within 5-95\% of that user's previous timings. So they weren't faster here, and it's unclear whether they were faster in the augmented condition either.

Justification about the electrode location (Cz): why the main reference throughout is Cz. The paper mentions this electrode earlier together with C3 and C4 as commonly used, but how does Cz stand out from the three?

** R2 has some suggestions about clarifying the definition of action augmentation and providing a brief explanation of how the present study achieves that much earlier in the paper can improve the readability and clarity. A good suggestion also is the inclusion of references to different states (idle and pre-movement state.

I urge the authors to read the reviewers' comments in detail to improve their work in the revision. I believe these can be fixed within the CHI timeline for R&R.


--