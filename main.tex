\documentclass[manuscript, screen=true, review=true, anonymous=true, authordraft=False]{acmart}
% \documentclass[sigconf]{acmart}

%%% START ADDITIONAL PACKAGES AND COMMANDS %%% 
\renewcommand{\sectionautorefname}{Section}
\renewcommand{\subsectionautorefname}{Section}
\renewcommand{\subsubsectionautorefname}{Section}

\usepackage{tabularx}
\usepackage{dcolumn} %Aligning numbers by decimal points in table columns
\newcolumntype{d}[1]{D{.}{.}{#1}}

\usepackage{subcaption}

\usepackage{todonotes}
\let\xtodo\todo
\renewcommand{\todo}[1]{\xtodo[inline,color=green!50]{#1}}
\newcommand{\itodo}[1]{\xtodo[inline]{#1}}
\newcommand{\red}[1]{\textcolor{red}{#1}}
\newcommand{\sven}[1]{\xtodo[inline,color=yellow!50]{Sven: #1}}
%%% END ADDITIONAL PACKAGES AND COMMANDS %%% 

%%
%% \BibTeX command to typeset BibTeX logo in the docs
\AtBeginDocument{%
  \providecommand\BibTeX{{%
    Bib\TeX}}}

%% Rights management information.  This information is sent to you
%% when you complete the rights form.  These commands have SAMPLE
%% values in them; it is your responsibility as an author to replace
%% the commands and values with those provided to you when you
%% complete the rights form.
\setcopyright{acmcopyright}
\copyrightyear{2018}
\acmYear{2018}
\acmDOI{XXXXXXX.XXXXXXX}

%% These commands are for a PROCEEDINGS abstract or paper.
\acmConference[Conference acronym 'XX]{Make sure to enter the correct
  conference title from your rights confirmation emai}{June 03--05,
  2018}{Woodstock, NY}
\acmPrice{15.00}
\acmISBN{978-1-4503-XXXX-X/18/06}


%%
%% Submission ID.
%% Use this when submitting an article to a sponsored event. You'll
%% receive a unique submission ID from the organizers
%% of the event, and this ID should be used as the parameter to this command.
%%\acmSubmissionID{123-A56-BU3}

%% A "teaser" image appears between the author and affiliation
%% information and the body of the document, and typically spans the
%% page.
% 
% This is what most people (90+%) will see from your paper!

% - [ ]  use hemingwayapp.com (+SEO if hardcore) to make figure captions super precise
% - [ ]  are all the axes labels readable, font size 10 or up
% - [ ]  check for missing information: give someone the figure with captions to someone neutral to check if everything is clear from the figure alone
% - [ ]  for digital version: what is alt-text of figure, i.e. when hovering with the mouse over the figure what text appears next to the cursor
% - [ ]  for accessibility: write figure caption for the blind
%     - [ ]  explain what the figure shows
%     - [ ]  refactor using hemingwayapp.com

\begin{teaserfigure}
  \includegraphics[width=\textwidth]{figures/task_design.png}
  \caption{(a) description of the experimental conditions, (b) their hypothesized effects on sense of agency from the perspective of a comparator model, is inspired by~\citet{Dewey2019-dg} (c) depicts the flow of movement generation in the participant and the system between conditions.}
  \Description{(a) description of the experimental conditions, (b) their hypothesized effects on sense of agency from the perspective of a comparator model, is inspired by~\citet{Dewey2019-dg} (c) depicts the flow of movement generation in the participant and the system between conditions.}
  \label{fig:teaser}
\end{teaserfigure}


%%
%% end of the preamble, start of the body of the document source.
\begin{document}

%%
%% The "title" command has an optional parameter,
%% allowing the author to define a "short title" to be used in page headers.
\title{% Preserving Agency in Human Augmentation using Brain Signals reflecting the Intent to Interact

% Preserving Agency in Action Augmentation using Brain Signals reflecting the Intent to Interact

% Intended Augmentation: Preserving Agency for Augmented Humans using Brain Signals reflecting the Intent to Interact

% Preserving Agency for Augmented Humans using Brain Signals reflecting the Intent to Interact

% Preserving Agency During Action Augmentation using Brain Signals of the Intent to Interact

% AAA: Agency-preserving Action Augmentation using Brain-Computer Interfaces
% Brain Signals of the Intent to Interact

% Agency-preserving Human Augmentation using Brain-Computer Interfaces

Agency-preserving Action Augmentation using Brain-Computer Interfaces

%Agency-preserving Action Augmentation using Brain Signals of the Intent to Interact

% agencEEG

% staying in the driving seat of augmented actions}

%%
%% The "author" command and its associated commands are used to define
%% the authors and their affiliations.
%% Of note is the shared affiliation of the first two authors, and the
%% "authornote" and "authornotemark" commands
%% used to denote shared contribution to the research.

\settopmatter{authorsperrow=3}

\author{Lukas Gehrke}
\orcid{0000-0003-3661-1973}
\affiliation{%
  \institution{TU Berlin}
  \city{Berlin}
  \postcode{10623}
  \country{Germany}
}
\email{lukas.gehrke@tu-berlin.de}

\author{Leonie Terfurth}
\orcid{}
\affiliation{%
  \institution{TU Berlin}
  \city{Berlin}
  \postcode{10623}
  \country{Germany}
}
\email{leonie.terfurth@tu-berlin.de}

\author{Klaus Gramann}
\orcid{0000-0003-2673-1832}
\affiliation{%
  \institution{TU Berlin}
  \city{Berlin}
  \postcode{10623}
  \country{Germany}
}
\email{klaus.gramann@tu-berlin.de}

%%
%% By default, the full list of authors will be used in the page
%% headers. Often, this list is too long, and will overlap
%% other information printed in the page headers. This command allows
%% the author to define a more concise list
%% of authors' names for this purpose.
\renewcommand{\shortauthors}{Gehrke et al.}

%%
%% The abstract is a short summary of the work to be presented in the
%% article.
\begin{abstract}

% Besides the title, this is what most people (90+%) will read from your paper!

% - [ ] Therefore, improve SEO (Search engine optimization): copy the abstract text to hemingwayapp.com to improve information density, keyword frequency and readability! Go through each sentence to shorten it and remove unnecessary words.

abstract text there only

\todo{What is the specific problem addressed?}
\todo{What have you done?}
\todo{What did you find out?}
\todo{What are the implications on a larger scale?}
\end{abstract}

%%
%% The code below is generated by the tool at http://dl.acm.org/ccs.cfm.
%% Please copy and paste the code instead of the example below.
%%

\begin{CCSXML}
<ccs2012>
    <concept_id>10003120.10003121.10003128</concept_id>
        <concept_desc>Human-centered computing~Human computer interaction (HCI)</concept_desc>
        <concept_significance>300</concept_significance>
    </concept>
 </ccs2012>
\end{CCSXML}
\ccsdesc[500]{Human-centered computing~Human computer interaction (HCI)}


%%
%% Keywords. The author(s) should pick words that accurately describe
%% the work being presented. Separate the keywords with commas.
\keywords{action augmentation, sense of agency, brain-computer interface, human-computer interaction, EEG, EMS, intention}

%%
%% This command processes the author and affiliation and title
%% information and builds the first part of the formatted document.
\maketitle

\section{Introduction}
% 1. Describe the Current Situtation
% Functions as a starting point and a common basis. Therefore it primarily contains recognizable and agreed points.
Advances in hardware that augment a user's physical actions have reignited dreams of overcoming human limitations, recovering lost abilities and simplifying skill acquisition~\citep{Goto2020-mw, Kunze2017-co}. \textcolor{green}{These technological advances include the miniaturization of the actuating hardware to wearable form factors and the direct sensing and stimulation capabilities of neural interfaces. Especially due to these characteristics, recent perspectives promote a change of the computing era from human-computer \textit{interaction} to \textit{integration}~\cite{Mueller2020-dl}. One key change in perspective is, that \textit{integrated} user's share agency with the computing machinery to execute tasks. This is a critical distinction as \textit{integration} technologies are designed to directly influence people’s bodies, their actions, and the resulting action outcomes.}

\textcolor{green}{Besides body and outcome augmentation, action augmentation has been defined as the case where a ``system assists the user’s action to produce the intended outcome.''~\cite{Cornelio2022-aq}. While such action augmentations can be realized purely on a software integration level, for example by an AI pair programmer, when it is designed to happen on a hardware level, further challenges emerge, specifically due to shared agency, which in this case means handing over control of ones body. Unfortunately,} such augmented users often report dissociative experiences, frequently disrupting their sense of agency (SoA)~\citep{Gilbert2017-ze, Gilbert2019-uc}.

% 2. What is the complication, challenge identified
% Spells the reason for acting now. It contains threats / opportunities and the hurdles that need to be overcome.
Having a SoA means experiencing control of our own voluntary actions, instead of them feeling as randomly happening to us. It has been shown that users are more likely to feel engaged and satisfied with an interaction, and are more likely to trust a system the more they experience SoA~\citep{Berberian2012-do, Miller2007-rb}. Hence, a key challenge to drive the adoption of human action augmentation then is to design for agency experience, so users feel as though they are in the ``driver's seat'' once again.

% 3. Question
% Asks the question how the hurdles of the Complication can be overcome. How can we prevent the threat or seize the opportunity? Also, what would be the benefits if the complication would be overcome?
% What have others done to address this and what do we propose "instead"
\textcolor{red}{One goal of green `on-body' action augmentation is to increase reaction times, thereby making an interaction more efficient. Previous work on such augmentation scenarios has shown that an actuation preceding a movement by about 80 ms preserves agency~\cite{Kasahara2019-sk}. Evidence from cognitive neuroscience indicates that from around 200ms before a voluntary movement, users are unable to ``veto'' their self-initiated movement~\cite{Schultze-Kraft2016-bx}. Here, the \textit{key} aspect for SoA in action augmentation becomes apparent: External influences on the user's body need to be in line with the user's intention.}

\textcolor{red}{In experimental settings, this is typically achieved by using strongly controlled `stimulus-response' paradigms. For example, reaction times when tapping on a touchscreen in response to a presented stimulus on screen, can be reduced by an augmentation device while maintaining SoA for the user~\cite{Kasahara2019-sk, Kasahara2021-gy}. In such controlled scenarios, the timing of the action augmentation can be tuned to be near optimal because the user behavior can be predicted with very high certainty to follow a previously presented stimulus. However, a key challenge remaining is to design systems that maintain agency during unpredictable actions of the user and \textit{without} control over the environment. In other words, how can we design a closed-loop system to deliver a \textit{natural} agency experience for users' augmented actions?}

% [4. (short) Answer Teaser
% Provides the answer on how to overcome the hurdles. Explains how this will help deflect the threats or seize the opportunities.
% keep this very short]

\subsection{Preserving Agency using Brain Signals of the Intent to (Inter-) act}
% 4. (long) Anwser with teaser image
% Provides the answer on how to overcome the hurdles. Explains how this will help deflect the threats or seize the opportunities.
In this paper, we present a prototype to overcome the disruption in experiencing agency during action augmentation. We developed a brain-computer interface (BCI) that establishes a fast communication channel between a user's brain signals and a physical end effector. The augmentation system (from hereon referred to simply as \textit{system}) controls the user's muscles at the time of their \textit{intent to interact}, as measured through readiness potentials (RP) manifesting in the user's electroencephalogram (EEG)~\cite{Schurger2021-vp, Schultze-Kraft2016-bx, Schultze-Kraft2021-cu}. 

\begin{figure}[!h]
    \centering
    \includegraphics[width=\columnwidth]{figures/teaser_new.png}%{figures/bci_game.png}
    \caption{Our augmentation system: When participants feel the spontaneous urge to move, readiness potentials (RPs) are picked up in the user's brain signals. A brain-computer interface (BCI) predicts the data to be in either of two classes: Idle or pre-movement. In the latter case, electrical muscle stimulation (EMS) is triggered and the user's hand is moved. Image taken with consent from participant.}
\end{figure}

To control the interaction in real-time, an RP-based classifier distinguished between two user states: \textit{idle}, reflecting the absence of an intent to act, and \textit{pre-movement}, indicating the presence of an intent to act.\textcolor{red}{When the system predicted a \textit{pre-movement} state, the user's action was augmented, potentially even before their own voluntary motor command. In our user study, we then applied a mixed-methods research approach to investigate whether keeping the physical impact on the user's body in line with their intention to move preserves their SoA.} \textcolor{green}{During \textit{idle}, participants were passively looking at a fixation cross. Instead, during \textit{pre-movement}, participants were instructed to voluntarily initiate a tap on a touchscreen whenever they felt the urge to do so. Previous work has indicated that the RP emerges during formation of conscious intention and is specific to voluntary action~\cite{Schultze-Kraft2020-rm, Travers2020-hf, Pares-Pujolras2019-ll}.}

\textcolor{green}{Upon predicting a \textit{pre-movement} state, the system augmented the user's action, potentially preceding their voluntary motor command. This augmentation moved the ring finger in accordance with the user's intention to act. We achieved the movement by leveraging electrical muscle stimulation (EMS) applied to the user's forearm flexor muscle. In our user study, a mixed-methods research approach was employed to explore whether aligning the physical impact on the user's body with their intention to move preserved their Sense of Agency (SoA).}



%%% Resources
\begin{comment}
% TODO merge this with the figure below
% \missingfigure{maybe small infographic here showing the connection from EEG volitional RP thought to EMS trigger on the Arm. <- better use this for the teaser image. Here better present the main results, i.e. SoA scores? -> Klaus comment: yes, but they don't have to be exclusive. The general setup with the LRP could appear with the main results both in the teaser or here? But the teaser would be more a visualization of the principle, no? That does not necessarily require real results but could show a hyperplane and data dots for examplification.}
% make this as an infographic with comic style drawing of augmented user and then the three pathways drawn
% \begin{figure}
%     \centering
%     \includegraphics[width=\columnwidth]{figures/setup_conditions_draft_small.png}
%     \caption{draft apperatus and conditions}
%     \label{fig:setup}
% \end{figure}
% \begin{figure}[!h]
%     \centering
%     \includegraphics[width=\columnwidth]{figures/augmented_com_model_colour.png}
%     \caption{draft - visualization of comparator model of the SoA in the case of augmented interactions (inspired by Dewey 2019)}
%     \label{fig:com_model}
% \end{figure}

% add results and general outcomes here?

% for now moved to introduction section
% We follow classical work in the neuroscience describing SoA as a bias in the perception of action \textit{outcome}: Intentional binding paradigms state that when a button press is followed by a -- delayed -- outcome, participants mentally compress the delay and reproduce it as such. Critically, this temporal compression only occurs following endogenous movement intention. The action outcome is mentally \textit{binded} to the intention. To reduce uncertainty about the binding, the brain `explains away' the excess delta, compressing the action-outcome delay <REF>.

% We found ... rendering the augmentation a \textit{natural}, integrated expansion of the user's body.



% By controlling the stimulus presentation, the experimenter can control the timing of the \textit{pre-emptive} gain of the user's action augmentation. 

% Resources
% The authors found a relation between the experience of agency and the \textit{pre-emptive} gain, i.e., how much earlier the action of the user could be elicited. However, setting this gain factor only works in stimulus-response scenarios where the user behavior can be predicted with high accuracy. 

% At around 80 ms preceding the naturally occurring action, user's integrate the augmentation and maintain an experience of agency. 

% if something happens for you in an interaction there is a cost to that:
% - impacts SoA which in turn may makes user's engage less -> citation?
% - may impact precision and accuracy, which can be very critical in high stakes scenarios, e.g. air traffic control -> citation?

\end{comment}
\section{Related Work}
%intro
In a simplified world, intentional movements directly leading to external events provide a strong sense of responsibility and control over the action's outcome. For instance, when we press a piano key, it is unequivocally clear that we are the ones performing and causing the action. However, the introduction of technology can complicate this relationship. Imagine playing a computer game where a key press controls an avatar playing the piano. The avatar's movement is now generated by a computer, blurring the lines between the performer and the cause of the action. Despite this complexity, research (sources to be provided, such as wn and Imamizu) suggests that as long as the button press and the avatar's movement remain consistent, we still feel a sense of control.

This lingering sense of control might be due to our intentions being reflected in the resulting action (sources needed). However, in augmented interactions, the dynamics change. When technology moves our fingers, our body performs the action, but we may experience a diminished feeling of causing the action [sources - augmented actions lead to less agency; check Yoshi und Harrard 2013, Moore, Wegner und Harrard 2009, Haggard, Clark 2002].

% why is feeling of control important
Because the feeling of control is related to (check for sources on learning effects etc.), understanding the reason for its loss in augmented interactions is relevant. 

%theorie 
Sense of agency is simplictily explained within the framework of motor control (see Frith and collegus 2000). When we perform an action without any external interference, the sensory consequences of that action align with the predicted sensory consequences, which are based on a copy of motor commands (i.e., efference copy). However, in situations where the movement is elicited by an external cause, as in augmented actions, the predicted sensory consequences may differ or be absent altogether. In such cases, the actual sensory feedback does not match the predicted sensory feedback.

According to this account, the feeling of agency arises only when the predicted sensory feedback perfectly corresponds to the actual sensory feedback. In other words, when our actions and their outcomes align seamlessly, we experience a sense of agency over those actions.

\begin{figure}[h]
  \centering
  \includegraphics[]{figures}
   \caption{}
  \Description{
}
  \label{fig:com_model}
\end{figure}
(see Carruthers 2012 for a summary of arguments against this model)



% In this paper, we present a physical action augmentation prototype and a user study to investigate the experience of agency during the use of the prototype. Hence, our research builds on previous qualitative and quantitative research on the sense of agency.
% on actuated haptic systems, brain-computer interfacing as well as
% moving the finger of a user when the user intends to do so, but before the user directs their muscles to move.

%Our research builds on previous engineering work in brain-computer interfacing and on theories of the sense of agency as discussed in cognitive science and neuroscience.


\subsection{The sense of agency in human-computer interaction}
% LT to write first version
\begin{comment}


% Sense of agency in augmented interaction 

- intention / voluntary movement is important for sense of agency
- involuntary movements produce less binding than do voluntary actions, or even reverse the effect entirely (Yoshi und Harrard 2013, Moore, Wegner und Harrard 2009)
it has been shown, that mere peripheral body movements, elicited by TMS simulations, produce a perceptual repulsion opposite to intentional binding in truly operant intentional actions (Haggard, Clark 2002)

There is an ongoing debate, whether the temporal attraction is specific to intentional movements, or is more generally related to the perception between action and outcome (Buhner 2012)
--> they suggest intentional action is not necessary for temporal binding, but that the binding results from a causal relation linking actions with consequences -- more general causal binding -- do we need to go there?

One idea: - reach "we-mode" in human-machine joint control ---> decode human intention (Zander 2011, Felke 2019, Shiskin 2016)
- maintain sense of agency in augmented interactions 
- doing this by leveraging action intention in brain



Experiencing control is an important factor in human-computer interaction (see Shneiderman and Nielsen). 

%%%%%%%%%% theory on sense of agency  / neural basis of sense of agency

The sense of agency, or this being in the "driving seat when it comes to our own actions"~\cite{Moore2016-ub} can be subdivided into the \textit{feeling} of agency, a low-level pre-reflective sensory process, and the \textit{judgement} of agency, a higher level reflective cognitive process, ~\cite{Moore2016-ub, Danry2022-xk}.

Agency is largely explained with a comparator model, describing internal computational mechanisms of human action control 
-- comparator model (Blakemore 2002, Frith 2000, Frith 2005)
-- integrate predictive coding??

- In the words of Harrard: making an association between action and outcome (Intention - Volition-Movement-Agency -Effect) --> find the source; is from yt video

- Looking at the sense of agency from the perspective of the comparator model
    - if prediction and sensory feedback match = we perceive sense of agency - not only for the big loop with sensory feedback (beep), but also for the forward loop (smaller) --> I can still think that I turned the light on, even if it takes 10 sec after pushing the switch, cause movement still worked as predicted

- In Augmented movements different (explain)
- Before deep diving in these results - excurse on how to measure agency

\subsection{Measuring agency experience}



% explicit measures - reflective (short paragraph)
% - questionnaires

%% from bergström 22
- read 18, 37 from  Bergstöem 
- 14,14,19.31,39,41,48
11,20,33

Sense of control is mostly measured with simple items with ratings like "It felt like I was in control of the hand I was looking at” rated on a 7-point Likert scale, +3 indicating strong agreement and −3 indicating strong disagreement (Longo and Haggard) or  "Indicate how much it felt like moving the joystick caused the object on the computer screen to move” as a measure of explicit sense of agency, also rated on a 7-point scale" (11) 
item asking participants to indicate the degree of control they have felt over the changes on the screen [1]. We used this item also as an agency measure (1) 

Researchers use implicit and explicit methods. Explicit methods involve directly asking participants to report their subjective experiences using items rated on Likert scales or percentages. These items, such as "It felt like I was in control of the hand I was looking at" (Longo & Haggard) and "Indicate how much it felt like moving the joystick caused the object on the computer screen to move" (Ebert & Wegner), may focus on either the action element or the outcome element of the sense of agency (see Moore).



( van der Wel et al., 2012; Chambon et al., 2013),“ 
Individuals’ sense of agency can be estimated with explicit self-reported measures using Likert’s scales or percentages ( van der Wel et al., 2012; Chambon et al., 2013),“ 

Items such as ""It felt like I was in control of the hand I was looking at” (longo Haggard) and "Indicate how much it felt like moving the joystick caused the object on the computer screen to move” (Ebert und Wegner). Notably, such items differ between focusing either on the action element or the outcome element (See MOore)



% - phenomenological interview

% implicit measures - pre-reflective (longer paragraph)
-subjective measures address a high level-SoA  based on subjective judgments of feeling of control (Barlas and Kopp)
- proxy for the low-level SoA as they do not require conscious reflection on one’s SoA (Synofzik et al., 2008a; Desantis et al., 2011; Moore and Obhi, 2012; Barlas and Obhi, 2014).“ ([Barlas und Kopp, 2018, p. 2]
- one such measure is intentional binding

% - intentional binding: explain what this is and give a bit more detail about the underlying mechanisms and ideas about whats going on in the brain here
- „refers to the perceived temporal attraction between voluntary actions and their outcomes (Haggard et al., 2002). More clearly, the temporal interval between actions and outcomes is perceived as shorter when these outcomes are produced by voluntary actions compared to when they follow, for instance, involuntary movements or external causes (Haggard et al., 2002).“ ([Barlas und Kopp, 2018, p. 2]
- While details about the relationship between measures of intentional binding and SoA are far from clear, they are extensivly used 
- alternatives to intentional binding - sensory attenuation / visual attention




\textit{Intentional binding} is the phenomenon of subjectively compressing the time interval between a voluntary action and the sensory consequences of that action~\cite{Moore2012-ic}.
- comparisons between active and passive finger tab 
    Kühn 2013 + Engbert 2007 : experimenter presses the finger down in a passive condition
    
    
When a voluntary action is causally linked with a sensory outcome, the action and its consequent effect are perceived as being closer together in time. This effect is called intentional binding.“ (from Jo 2014)

- Read More 2009 (interval estimations were shorter (i.e., stronger binding) in voluntary than involuntary movements) 
- add figuure action binding and effect binding (e.g.  in WEN and Imamizu)


!! new paper suggesting that temporal binding / intentional binding is not a measure of sense of agency !!(Gutzeit!)  


% connection between implicit and explicit measures

- intentional binding effects are sometimes discrepant from explicit judgment of agency (see 30-33 in Wen and Imamizu) (Saito 2015; Majchrowitz 2018; Ebert
- Intentional binding is e.g. influenced by causal relations between events, attention, and arousal (see Wen and Imamizu)



% Sense of agency in augmented interaction 

- intention / voluntary movement is important for sense of agency
- involuntary movements produce less binding than do voluntary actions, or even reverse the effect entirely (Yoshi und Harrard 2013, Moore, Wegner und Harrard 2009)
it has been shown, that mere peripheral body movements, elicited by TMS simulations, produce a perceptual repulsion opposite to intentional binding in truly operant intentional actions (Haggard, Clark 2002)

There is an ongoing debate, whether the temporal attraction is specific to intentional movements, or is more generally related to the perception between action and outcome (Buhner 2012)
--> they suggest intentional action is not necessary for temporal binding, but that the binding results from a causal relation linking actions with consequences -- more general causal binding -- do we need to go there?

One idea: - reach "we-mode" in human-machine joint control ---> decode human intention (Zander 2011, Felke 2019, Shiskin 2016)
- maintain sense of agency in augmented interactions 
- doing this by leveraging action intention in brain




% RESEARCH GAP




\subsection{Sense of agency and actuated haptic systems}


\ subsection{Movement Intention}
idea: when we now compare completely voluntary movements with movements triggerd by the brain signal of our intention to move - what happens?

% close this paragraph with the problem that of the control mechanism of the actuator hardware and the disruotion in agency

% PosssessedHand, Gilbert-I miss being me, 


%%%%%%%%%%%%%%%%%%%%%%
Other
- EEG patterns related to readiness potential and intentional  binding sometimes show disceprant patterns ( see WEN and Imamizu 38 und 39) (wittmann 2014, Goldberg 2017)
\end{comment}
% \section{Agency Prototype}
Here we present a prototype to overcome the disruption in experiencing agency during action augmentation. Our proposed system controls an end-effector, here EMS, by leveraging EEG signals reflecting the intent to (inter)act. To control the interaction in real-time, an EEG classifier is used to discriminate between two user states: either the user is predicted to be in an \textit{idle} state, reflecting the absence of an intent to act-- or a \textit{pre-movement} state, indicating the presence of an intent to act. When the system predicts a \textit{pre-movement} state, EMS is triggered in real-time, augmenting the user's action even before their own voluntary motor command.

\subsection{Controlling actuated haptic experiences using brain signals reflecting the intent to interact}
% alternative headings:
% Triggering EMS using EEG Signals

To discriminate between the two \textit{action} states, we leveraged the EEG Readiness Potential (RP), or \textit{lateralized readiness potential}. It is an amplitude fluctuation that has frequently been observed preceding voluntary action~\cite{Deecke1969-bl, Libet1983-qu}. The RP is reliably observed at electrodes placed over the motor cortex contralateral to the acting hand. Since its measurable onset precedes the time of participants self-reported conscious movement intention, it has drawn much interest with respect to the debate on free will, see~\cite{Schurger2021-vp} for a recent neuroscientific perspective. However, evidence abounds for its role in action preparation. A RP is typically comprised of two stages: an early slow stage that begins up to two seconds before the actual movement and a late steep stage that starts about 400 milliseconds before movement. The first stage manifests in the pre-supplementary motor area and transfers to the premotor cortex shortly after. The second stage manifests contra-laterally in the primary motor cortex~\cite{Shibasaki2006-mt}. A recent study has shown that the RP is heavily ingrained in the subconscious mechanisms preceding movements that people cannot explicitly suppress~\cite{Schultze-Kraft2021-cu}. In their study,~\cite{Schultze-Kraft2021-cu} asked participants to find a way to perform voluntary movements while keeping accompanying RP amplitudes as small as possible. After each trial they informed participants about the strength of the RP in the current trial, so participants had a feedback metric to optimize for. They found participants unable to suppress their RP. This inability to suppress the RP renders it a reliable feature for classification.

\subsection{Prototype components}
Our system, depicted in Figure~\ref{}, comprised: (1) a 64-channel EEG system, (2) a 1 channel Electromyography (EMG) system, (3) a simple Arduino controlled switch, and (4) a medically-compliant EMS device connected via two electrodes worn on the arm. More information on the specific hardware specifications is given in the methods and user study sections below. To assist readers in replicating our prototype, we provide the necessary technical details and source code for data processing, classifier training, and real-time application\footnote{[anonymized]}. We used LSL\footnote{https://github.com/sccn/labstreaminglayer} to make the EEG and EMG data streams available in the network and synchronize the recordings of EEG data, motion capture, and an experiment marker stream that marked sections of the study procedure.

\missingfigure{show participant with measurement system and description next to each element}

\subsection{Brain-Computer Interface}
% alternative headings:
% Brain-Computer Interface to Detect EEG Readiness Potentials }
% Classifiying EEG Signals of the Intent to Interact

The presented classification scheme required individual training data per participant, as transfer learning remains challenging with EEG~\cite{Wan2021-zz}. Therefore, we initially obtained training data on a variant of our task that did not include any muscle stimulation. In our task, participants were instructed, following an initial waiting period, to tap a touchscreen at their own volition with their ring finger, see figure ~\ref{fig:teaser}. 

\subsubsection{Labelling movement classes using EMG}
To obtain behavioral labels at a high temporal resolution, we leveraged EMG from the flexor digitorum profundus to detect finger movement onsets of the ring finger. Subsequently the data were labelled into two classes: an \textit{idle} class representing resting background activity and a \textit{pre-movement} class, representing the intent to (inter-)act. To this end, EMG amplitudes of the second preceding the screen tap were band-pass filtered from 20 to 100 Hz and subsequently squared. Next, noisy data segments were determined. A segment was rejected as an outlier using Matlab's \textit{isoutlier} function using default parameters. To label the time of movement onset, the first sample where the amplitude exceeded the 95th percentile of the data epoch was selected. The two event classes were then defined as follows: \textit{pre-movement} from -1000 to 0 ms preceding the movement onsets and \textit{idle}, a one second data segment between trials where participants were looking at a fixation cross.
% from isoutlier documentation (mean segment amplitude exceeding 3 scaled median absolute deviation)

% \missingfigure{some picture from the setup and prototype}

\subsubsection{Selecting discriminative EEG channels}
In a first step to prepare the EEG data for classification, noisy data segments were rejected using the identical method as for rejection of EMG segments described above. Then, to decrease dimensionality of the EEG data, a selection of discriminative channels was conducted following an approach outlined by~\citep{Schultze-Kraft2021-cu} albeit with slight modifications.

First, for each channel and both \textit{idle} and \textit{pre-movement} class, the mean slope of the amplitude was obtained by fitting a linear regression, indicating how much the signal had changed in the course of the 1s segments. In line with the literature on the RP, there should be a change over time in the \textit{pre-movement} segment but not in the \textit{idle} segment. Then, all channels were sorted (1) in descending order by slope in the \textit{pre-movement} segments and (2) in ascending order by slope in the \textit{idle} segments. The idea was, that an ideal channel shows a strong change in signal over the course of a \textit{pre-movement} segment but shows no difference during a \textit{idle} segment. The ranks were joined and the 10 highest-ranking channels were selected. In any case, channels C3, C4, and Cz were included in the 10 selected channels for every participant even if they were not within the 10 highest ranks as these are most frequently reported in studies on the RP.

\subsubsection{Training of EEG classifier}
To classify EEG data a regularized linear discriminant analysis (LDA) was trained for each participant individually. The EEG classifier was then used in the test phase to detect RPs and trigger EMS. The classifier was trained using the slopes of the \textit{idle} and \textit{pre-movement} segments of the 10 most discriminative channels as features, hence comprising a 10 dimensional feature vector.
\todo{extend this after settling on a final feature set}

\subsubsection{Real-time application and EMS control}
During real-time application, the EEG data was buffered for the last second for the selected channels. The slope features were then computed and applied to the model. An Arduino was used to flip a switch whenever the EEG classifier predicted the \textit{pre-movement} class with an above .7 probability, thereby triggering the attached actuating hardware. 
\todo{this may still change, adapt accordingly later}
In our current prototype, we used a medically-compliant EMS device. We constrained the times when the switch could be flipped to specific moments in the experiment where participants were getting ready to perform an action, thereby limiting the impact of false-positive classification. 
% This guaranteed that participants experienced EMS only during movement preparation.


%%%% Resources

% old version
% First, the 10 channels with the strongest discrimination between \textit{pre-movement} and \textit{idle} segments were selected with the following procedure: For both \textit{pre-movement} and \textit{idle} the mean signal in the last 100 ms of each segment was subtracted from the mean signal in the first 100ms of the segment. The resulting value thus indicated how much the signal had changed in the course of the 1s segments. In line with the literature there should be a change over time in the \textit{pre-movement} segment but not in the \textit{idle} segment. Then, all channels were sorted (1) in descending order by the signal difference in the \textit{pre-movement} segments and (2) in ascending order by the signal difference in the \textit{idle} segments. The idea was, that an ideal channel shows a strong change in signal over the course of a \textit{pre-movement} segment but shows no difference during a \textit{idle} segment. The ranks were joined and the 10 best channels were selected. In any case, channels C3, C4 and Cz were kept for every participant as these are most frequently reported in studies on the RP.
% An EEG classifier was trained on \textit{pre-movement} and \textit{idle} data segments using 10 features from the 20 most informative EEG channels. To classify single-trial motion and EEG data we used a regularized linear discriminant analysis (LDA) that was trained for each participant individually. 
% These were baseline corrected by subtracting the mean of the last 100 ms preceding the segment. The features were concatenated across all 20 selected channels and 10 time windows to obtain a 200 dimensional spatiotemporal feature vector per \textit{pre-movement} and \textit{idle} segment.

% During the test phase, both classifiers were applied to the real-time data. With the motion data streaming at 90 Hz we subjected the current sample to the motion classifier. For the EEG data streaming at 250 Hz, the EEG data of the last 1000ms was buffered for the selected best channels and 10 windowed means per channel were computed. Identical to the training data, these features were baseline corrected by subtracting the mean from the last 100 ms preceding the last second, i.e. -1100 ms to -1000 ms, resulting in a 200 dimensional baseline corrected spatiotemporal feature vector. Hence, both classifiers yielded one output value at each sample point.

\section{User study \& Methods}
With this study, we wanted to find out whether the experience of agency can be preserved during physical action augmentation when the user's own brain signals serve as the control signal to a physical end effector. To this end, we assessed whether our prototype preserved agency using a mixed methods approach including a psychometric test of intentional binding, one standardized question, and a qualitative (phenomenological) interview.
% todo add some more framing here to make it easier to step through the methods, maybe: For this paper we designed a prototype that lies between the two edge cases -- no intention and control to full control -- and investigate user's sense of agency using both explicit and implicit measures. 

\subsection{Participants}
Nine participants (M = 30.00 years, SD = 3.81) were recruited from our local institution and through the institute's online participant pool. All participants were right-handed. Participants were compensated with course credit or 12 Euro per hour of study participation. Prior to their participation, they were informed of the nature of the experiment, recording, and anonymization procedures and signed a consent form. The experiment was approved by the ethics committee of [anonymized]. One participant had to be excluded from further data analyses due to significant deviations from the instructions in the execution of the task.
%the Department of Psychology and Ergonomics at TU Berlin (tracking number: BPN\_GEH\_2\_230130220421).

\subsection{Experimental Task, Design and Procedure}
Participants performed a simple tapping task in three conditions with 75 trials each: BASELINE, NO INTENTION (to move) and INTENTION (to move) EMS, see figure~\ref{fig:task_design}.

\indent\textbf{BASELINE.} The task went as follows: (1) a fixation cross appeared on a tablet screen and participants were instructed to rest and wait until it disappeared; (2) then, they were instructed to wait for a brief moment (2 to 3s), before (3) initiating their movement and tap the screen. In line with the literature on the origin of the RP generating process they were told ``to avoid pre-planning the movement, avoid any obvious rhythm over trials, and to press when they felt the spontaneous urge to move''~\cite{Schultze-Kraft2021-cu}. (4) After the screen was tapped, a tone was played at a pseudo-random delay of 200, 350, or 500ms. Participants were now asked to estimate the delay, typing in their answer on a number pad of an attached keyboard. After confirming their answer by hitting the return key, the next trial started.
During BASELINE, participants were equipped with Electromyogram (EMG) sensors instead of EMS electrodes. BASELINE data was used as training data for the BCI, see section~\ref{BCI} below, as well as to select stereotypical reaction times in the NO INTENTION condition. 

\indent\textbf{NO INTENTION.} The task structure was identical to BASELINE, however, participants were now instructed to hold and wait for the muscle stimulation to move their finger thereby eliciting the screen tap. The timing for the EMS trigger was taken by randomly choosing a time between the 5th and 95th percentile of their actual individual reaction time in the BASELINE condition. 

\indent\textbf{INTENTION.} The task and instruction were identical to BASELINE with one additional instruction: ``you will now work \textit{with} the system''. During INTENTION, the muscle stimulation hardware was controlled by the BCI. The classifier was set to active after the fixation cross disappeared and until a screen tap was registered. Hence, the muscle was not stimulated at other times during a trial, so as not to interfere with participants typing in their time estimation response.

The order of the conditions was not pseudo-randomized since training data obtained in BASELINE was required for both NO INTENTION and INTENTION. Furthermore, INTENTION was always the last condition, thereby allowing for a prolonged interview.

\begin{figure*}
    \centering
    \includegraphics[width=\textwidth]{figures/task_design.png}
    \caption{(a) description of the experimental conditions, (b) their hypothesized effects on sense of agency from the perspective of a comparator model, is inspired by~\citet{Dewey2019-dg} (c) depicts the flow of movement generation in the participant and the system between conditions.}
    \label{fig:task_design}
\end{figure*}

\begin{figure}
    \centering
    \includegraphics[width=\columnwidth]{figures/task_progression.png}
    \caption{Visualization of task progression}
    \label{fig:progression}
\end{figure}

\subsection{Apparatus}
The experimental setup, depicted in Figure~\ref{}, comprised: (1) a 1-channel EMG device, (2) a 64-channel EEG system, (3) a medically-compliant EMS device connected via two electrodes worn on the forearm, and (4) a tablet to run the experiment and collect behavioral responses. To assist readers in replicating our experiment, we provide the necessary technical details, the complete source code for the experiment, the collected data, and the analysis scripts\footnote{[anonymized]}.
% TODO add links after anonymization is lifted.

\indent\textbf{(1) EMG Recording.} EMG data was recorded from 1 bipolar channel using the BrainAmp ExG amplifier (BrainProducts GmbH, Gilching, Germany). The two electrodes were placed above the flexor digitorum profundus with a reference electrode located on the wrist bone. EMG data was collected in synchrony with the EEG data through BrainProducts BrainVision Recorder.

\indent\textbf{(2) EEG Recording.} EEG data was recorded from 64 actively amplified Ag/AgCl electrodes in an actiCap Snap cap using BrainAmp DC amplifiers from BrainProducts. Electrodes were placed according to the extended international 10–20 system \cite{Jasper1983-uw}. One electrode was placed under the right eye to provide additional information about eye movements (vEOG). After fitting the cap, all electrodes were filled with conductive gel to ensure proper conductivity, and electrode impedance was brought below 10$\Omega$. EEG (and EMG) data were recorded with a sampling rate of 250 Hz. 

We used LSL\footnote{https://github.com/sccn/labstreaminglayer} to make the data streams available in the network and synchronize the recordings of EEG/EMG data and an experiment marker stream that marked sections of the study procedure.

\indent\textbf{(3) Electrical Muscle Stimulation.} We actuated the ring finger via EMS, which was delivered with two electrodes attached to the participants' flexor digitorum profundus muscle. We utilized the flexor digitorum profundus since we found that we can robustly actuate it without inducing unintended motion of neighboring fingers. This finger actuation was achieved via a medically-compliant battery powered muscle stimulator (TENS/EMS Super Duo Plus, prorelax, Düren, Germany). The EMS system's output was controlled by flipping a solid state relay (silent) connected via an Arduino Uno (Arduino, Monza, Italy) to the experiment computer. The EMS was pre-calibrated by the participant to ensure a pain-free but effective stimulation and robust actuation leading to an `immediate' tap on the touchscreen after actuation. To ensure a comfortable experimental experience so that participants relaxed their arm musculature as much as possible, a custom built hand rest (support device) was placed on top of the touchscreen for participants, see figure ~\ref{fig:task_design} c.
% [todo] What were the settings of the stimulation?

\indent\textbf{(4) Experiment Presentation and Collection of Behavioral Responses.} An Acer Group (Acer Inc, Taipeh, Taiwan) tablet was used to present the task to participants and record their behavioral responses. In addition to the tablet, we used an external keyboard to allow response input of users indicating their timing judgements.

% alternative headings:
% Brain-Computer Interface to Detect EEG Readiness Potentials }
% Classifiying EEG Signals of the Intent to Interact
\subsection{Brain-Computer Interface}\label{BCI}
The data obtained in BASELINE was used to train the BCI. We utilized the [anonymized] and the EEGLAB~\cite{Delorme2004-sn} toolbox inside the MATLAB (The MathWorks Inc. Natick, MA, USA) environment for preprocessing both EEG and EMG data. First, to generate behavioral labels at a high temporal resolution for the EEG-based classifier, we leveraged EMG data from the flexor digitorum profundus. EMG amplitudes were band-pass filtered from 20 to 100 Hz and subsequently squared. Next, in order to label the time of movement onset, the EMG data was averaged across trials for the second preceding the screen tap. From this averaged data, the first sample where the EMG amplitude exceeded the 95th percentile was selected as the time of movement onset.
% bemobil pipeline

% TODO Next, noisy data segments were determined. 
% A segment was rejected as an outlier using Matlab's \textit{isoutlier} function using default parameters. % from isoutlier documentation (mean segment amplitude exceeding 3 scaled median absolute deviation)

Two event classes were then defined as follows: \textit{pre-movement} from -1000 to 0 ms preceding the (EMG detected) movement onsets and \textit{idle}, a one-second data segment between trials where participants were looking at a fixation cross.

\subsubsection{Preprocessing EEG and Selecting discriminative channels}
The EEG data was band pass filtered from 0.1 to 15 Hz. In the first step to prepare the EEG data for classification, noisy data segments were rejected. To this end, the EEGLAB function `autorej' was used, keeping default parameters. A trial was excluded if \textit{either} data from the \textit{idle} or \textit{pre-movement} class was rejected.

Then, to decrease the dimensionality of the EEG data, a selection of discriminative channels was conducted following a selection approach outlined by~\citep{Schultze-Kraft2021-cu}. First, for both \textit{pre-movement} and \textit{idle} the mean signal in the last 100 ms of each segment was subtracted from the mean signal in the first 100 ms of the segment. The resulting value thus indicated how much the signal had changed in the course of the 1 s segments. In line with the literature, there should be a change over time in the \textit{pre-movement} segment but not in the \textit{idle} segment. Then, all channels were sorted (1) in descending order by the signal difference in the \textit{pre-movement} segments and (2) in ascending order by the signal difference in the \textit{idle} segments. With respect to the ability of the system to discriminate between movement intention and the absence of movement intention, an ideal channel should show a strong change in signal over the course of a \textit{pre-movement} segment but no difference during an \textit{idle} segment. The ranks of the two criteria were joined by summation and the resulting order was saved for later use in training and real-time application of the classifier. Lastly, channels C3, C4 and Cz were moved to the top of the order for every participant as these are most frequently reported in studies on the RP.

\subsubsection{Training of EEG classifier}
To classify EEG data a linear discriminant analysis (LDA) with shrinkage regularization (automatic shrinkage using the Ledoit-Wolf lemma~\cite{Ledoit2004-bi}) was trained for each participant individually. As single-trial features, the (linear) slope coefficient was obtained for both \textit{idle} and \textit{pre-movement} segments by fitting a linear regression. The classifier was then trained using the slope coefficient feature of the selection of the most discriminative channels as features, hence in the end, generating a feature vector in the dimension of channels that were kept for classification. We purposefully constrained the dimensionality of the feature vector to avoid over-fitting. 

Using scikit-learn~\cite{Pedregosa2012-sj}, an LDA with automatic shrinkage was cross-validated (using 5-folds) for a grid search from 6 to 20 channels with a step size of adding 2 channels. Following this grid search, the cross-validation with the highest accuracy determined the number of channels that were kept for training the model for real-time application. Ultimately, to determine the threshold at which, during real-time application, the classifier would trigger the EMS, we computed the Receiver-operator characteristic (ROC) and from this selected the threshold at 15 \% false positive rate.

\subsubsection{Real-time application and EMS control}
During real-time application, the EEG data was buffered for the last second for the selected channels. The data was band-pass filtered analogously to the training data from 0.1 to 15 Hz. Next, the slope was computed for the discriminative channels selected in the classifier cross-validation grid search. This procedure ran at an update rate of 10 Hz, hence every 100 ms a new prediction was obtained from the classifier. To smooth the prediction output with the goal of reducing false predictions due to unlikely peaks, the predicted probability for the \textit{pre-movement} class was smoothed by averaging the last two predictions with a weighting (.3 and .5). Then, at 10Hz update rate, this smoothed probability, as well as the predicted class for the current frame were gating the EMS switch: when the probability exceeded the threshold and the currently predicted class was \textit{pre-movement}, the switch was opened for 0.5s.

% To trigger the EMS, an Arduino was used to flip a switch whenever triggering the attached EMS. We constrained the times when the switch could be flipped to specific moments in the experiment where participants were getting ready to perform an action, thereby limiting the impact of distracting EMS pulses at resting phases during the experimental trials. %This guaranteed that participants experienced EMS only during movement preparation

\subsection{Measures of Agency Experience: Intentional Binding, Question \& Interview}
To assess if SoA was preserved, an intentional binding measure, a standardized question, and an interview following the movement INTENTION block were investigated. At the conclusion of each condition we prompted participants to rate their experienced agency on a 7-point Likert scale with the statement ``It felt like I was in control of the movements during the task.'', the item was copied from~\cite{Hornbaek}. Following INTENTION we interviewed users about their experience working with the EMS BCI. After prompting users to recall their experience and summarize what their task had been we set the focus to the tapping movement and asked them to ignore the time estimation task for the following questions. We entered the open part of the interview by asking: ``What did the system do?'' followed up by ``What was the difference between the three conditions''. After some time, and depending on their answers, we reset the focus to the INTENTION condition and asked ``How often was the system active?'' followed by ``What do you think caused the actions of the system?''. This was then followed up by an `open' interview in which we frequently asked `how' and `why' questions to inquire about the user's experience.

% add back in after anonymization: All interviews were manually transcribed and then automatically translated using DeepL\footnote{https://www.deepl.com/translator}
We analyzed the interviews that always followed the INTENTION condition by loosely following~\citet{Mayring2015-pp}. All interviews were manually transcribed and [anonymized]. Prior to screening the texts, two experts clustered the responses into 3 clusters: First, ``Functionality'' referred to what participants attributed the source of the stimulation. Next, we clustered responses according to ``Guessed percentage'' of correct interaction, i.e. participants estimate of how well the system was aligned with their intention. For the cluster ``Correct Interaction'', we specifically queried participants to recall the moments where the stimulation felt in line with their intention to move, then we clustered their responses into positive and negative valence sentiments. 

\subsection{Statistical Analyses}
To confirm and demonstrate the discriminative power of the EEG features, we plotted the amplitude time course of electrode Cz between \textit{pre-move} and \textit{idle} epochs. Next, the slope coefficients were extracted and a paired t-test was conducted. Ultimately, to assess the classifier performance, we report the F1 score, i.e. the harmonic mean between of the precision and recall, and plot the ROC.

\subsubsection{Hypotheses Testing}
Prior to any statistical analyses of the intentional binding measure, outlier trials were rejected. We applied `extreme outlier removal' using Tukey's method~\cite{Tukey1949-sl}. Three time intervals were considered relevant to describe `regular' behavior across trials which would manifest by: (1) Tapping the screen in a reasonable interval after the fixation cross disappeared. An excessively short or long delay indicated that participants either tapped the screen prematurely by accident or they were checking in with the experimenter, respectively. (2) Providing a `reasonable' estimation in the intentional binding task. (3) The EMS stimulation leading to an \textit{immediate} screen tap. A long delay between the EMS trigger and the subsequent screen tap indicated that the stimulation was not strong enough in this trial to lead to muscle actuation resulting in a screen tap. Taken together, applying Tukey's methods to these criteria led to the exclusion of 102 trials across all participants (M = 34.00, SD = 16.46) and a final data set of 1698 trials.

In line with the literature on intentional binding, we hypothesized that for both BASELINE and INTENTION conditions the time intervals should be underestimated. This should not be the case for the augmented NO INTENTION condition where users had no intention to move. Hence, when binding occurs, the intervals should be underestimated, hinting at a higher SoA. To test this, we fit a linear mixed effects model with \textit{condition} (BASELINE, NO INTENTION, INTENTION) as a fixed effect. As a random effect, we considered the intercepts for participants. We obtained p-values of the coefficients by calculating likelihood-ratio tests~\cite{winterLinearModelsLinear2013}. All parameters were estimated by maximum likelihood estimation~\cite[see][]{Pinheiro2000}. Post-hoc, pairwise contrasts were computed using estimated marginal means with Tukey's methods for multiple comparisons corrections~\cite{Lenth2020-xk}.

Next, we hypothesized that subjective ratings of control over the tap movement are comparable between BASELINE and INTENTION and lower in the NO INTENTION condition. Again, we fit a linear mixed effects model with \textit{condition} as a fixed effect and participant ID as a random effect. Coefficients were assessed in the same way as for the intentional binding parameter above.

In short, we tested two main hypotheses with regard to experiencing agency in our three experimental conditions: (1) participants underestimate the tone delay when acting intentionally. Adding EMS in line with participants' intention to move does not affect this underestimation. (2) The subjective feeling of control is comparable between BASELINE and INTENTION conditions. NO INTENTION should decrease the feeling of control significantly. Additionally, we report the clustered interviews anecdotally.


%%% Resources
% An additional cluster ``takeaways'' was used to cluster other, miscellaneous,  interesting comments.

% "how did it feel like?". If user's made reference themselves to the EMS device we followed up by inquiring: "why do you think it (the EMS trigger) happened?", "What do you think triggered it?". The goal was to learn about the source attribution user's made about the EMS trigger. Since we leveraged the RP, we hypothesized that user's would attribute, at least in part, the EMS trigger to themselves, i.e. their movement intention. Next, we pressed deeper into learning about the cooperative nature between user and system. We asked "Do you believe you did it alone?" and when they made reference to a specific situation where the EMS was triggered we followed up with "What did it feel like, did you had the feeling the system controlled you or did you cooperate?" and "What would have to be there for it to feel like a cooperation?". We closed the interview with two additional exploratory questions: "Did the system influence your performance? If yes, how and why?" as well as asking them about any application scenarios they have in mind. 

% \indent\textbf{(1) Intentional Binding.}
% At the end of each trial participants were tasked to estimate the delay between the screen tap and a subsequent tone. The real delay was pseudo-randomized out of [200, 350 and 500] ms. With 75 trials per condition, each real delay occurred 25 times. 

% \indent\textbf{(2) Questionnaire.}

% \subsection{Brain-Computer Interface}
% The presented classification scheme required individual training data per participant, as transfer learning remains challenging with EEG~\cite{Wan2021-zz}. Therefore, we initially obtained training data on a variant of our task that did not include any muscle stimulation. In our task, participants were instructed, following an initial waiting period, to tap a touchscreen with their ring finger at their own volition, see figure ~\ref{fig:teaser}. 
% \subsubsection{Labelling movement classes using EMG}3

% Subsequently the data were labelled into two classes: an \textit{idle} class representing resting background activity and a \textit{pre-movement} class, representing the intent to (inter-)act.

% In summary, the experiment consisted of five phases: (1) a setup phase; (2) the BASELINE condition to obtain training data and labels; (3) the NO INTENTION condition: being passively moved by the EMS system; (4) the INTENTION EMS condition where the BCI was triggering the EMS system that in turn made participants tap the screen; and (5) an open interview to inquire about their experienced agency following the INTENTION EMS condition.

% \subsubsection{Experiment Design and Procedure}
% The first condition was used to obtain training data for the single-trial classification system. Here participants were wearing the EMG sensors instead of the EMS actuators and were instructed to tap the screen at their own volition (BASELINE, see instructions above). In the second condition, participants were instructed to hold and wait for the EMS system to move their finger thereby eliciting the screen tap (NO INTENTION). The timing for the EMS trigger was taken by randomly choosing a time between the 5th and 95th percentile of their actual reaction time in the BASELINE condition. In the third condition, the EMS trigger was controlled through the readiness potential classifier (INTENTION EMS). Participants were given the instruction: ``you will now work \textit{with} the system'' along the same instruction as in the BASELINE condition. The classifier was set to active after the fixation cross disappeared and until a screen tap was registered. Hence, the EMS was not triggered at other times during a trial, so as not to interfere with participants typing in their time estimation response.

% To not break the user's immersion by disturbing the flow of the interaction we decided against asking qualitative questions at the end of each and every trial.

% old:
% \subsubsection{Reproducing Results and Data Availability}
% Data, experimental protocol, analyses code including scripts for a reproduction of the presented results are hosted at open science foundation (OSF)\footnote{[anonymized]}. BIDS formatted data is hosted on openneuro~\cite{}.


% , and the interval between EMS stimulation and tap as fixed effects

% Hypothese:
% In line with the literature on intentional binding, we hypothesized this duration to remain stable in the case that agency was preserved during the test phase with EMS. A deviation of the tone duration from the training phase was hypothesized to reflect a disruption in experienced agency.

% operationalising temporal binding = underestimation (here or somewhere else) 

% \indent\textbf{(1) Hand Tracking.} We used an HTC Vive Tracker, attached to the participant's wrist, to track their right hand at 90Hz. Therefore, we ran a simple Unity3D scene and streamed the data via the labstreaminglatyer (LSL)\footnote{https://github.com/sccn/labstreaminglayer}

% At the end of each trial participants were tasked to compare the duration of their arm movement to a tone and indicate whether the tone was longer or shorter. We set the initial duration at 500 ms for each participant. When participants indicated the tone to be longer or shorter by pressing either the up or down key we added, or respectively subtracted, 100 ms from the tone duration. This new duration was then used for the next trial and so on. With 60 trials in the training phase, we hypothesized participants would oscillate 50-50 around a target duration after about 2/3 of the training trials. The final duration after all training trials was then used again as the start duration during the test phase. In line with the literature on intentional binding, we hypothesized this duration to remain stable in the case that agency was preserved during the test phase with EMS. A deviation of the tone duration from the training phase was hypothesized to reflect a disruption in experienced agency.


% (1) participants waited for a new letter to appear on screen; (2) then, they were instructed to wait for a brief moment (~2s), before (3) reaching out and pressing the key. In line with the literature on the origin of the RP generating process they were told "to avoid preplanning the movement, avoid any obvious rhythm, and to press when they felt the spontaneous urge to move". (4) Two \todo{how many, but do keep this constant in order for them to focus on the duration of their action} seconds after they pressed the key, they were presented with a tone and asked to judge whether their movement, from the start of their reach to the button press, lasted longer or shorter than the tone. (5) When they perceived the tone to be longer, they had to press the up key or vice versa the down key for perceiving that the tone was shorter than their movement. Following an inter-trial-interval the next trial started.

% For each trial, a key was randomly selected from one of two groups of letters: (1) 'a', 's', 'd' or (2) 'j', 'k', 'l'. The selected group was then changed with every trial. This was done to keep the task engaging while maintaining a comparable reaching movement between trials.

% ERP with significance mask for Cz? Assessing significance of classifier

% IB Task: ttest of final ib time between with and without EMS -> # subjects input values
% did they increase the IB task in more than 50% of the trials after EMS?

% questionnaires: ttest against zero to report direction

% interviews: anecdotal summary and sentiment analyses with two groups splitting questionnaires at 0

% correlation with EEG classifier accuracy?

% evaluation of EEG classifier: can report false negatives, how often was it missing to predict a movement onset? hand was moving without EMS being triggered occurred how often? Not possible to evlaute false positives as no ground truth available
\section{Results}
In the intentional binding task, i.e. the estimation of the time interval between tap and tone, participants generally underestimated the average real delay (350ms) in all three conditions (INTENTION M = 160.3, SD = 116; EXTERNAL M = 176.7, SD = 120.7, AUGMENTED M = 162.6, SD = 118.9). The condition influenced the estimation (${\chi^{2}_{(2)}} = 15.9, p < .001$). In EXTERNAL the underestimation was less pronounced compared to INTENTION ($beta = 19.2, p < 0.001$). This was also the case in EXTERNAL, with less pronounced underestimations of the time interval between tap and tone compared to AUGMENTED ($beta = 18.6, p < 0.001$). No difference was observed between INTENTION and AUGMENTED.

\begin{figure}
    \centering
    \includegraphics[width=\columnwidth]{figures/behavioral_results.pdf}
    \caption{(a) Temporal binding effect over conditions. With an average presented time interval of 350ms, lower values in the estimated time interval indicate an underestimation, i.e., temporal binding. (b) Subjective ratings of control across conditions. Significance labels obtained from post-hoc tests on estimated marginal means.}
    \label{fig:results}
\end{figure}

The subjective rating of control differed between conditions (${\chi^{2}_{(2)}} = 35.9, p < .001$). Post-hoc pairwise comparisons revealed that participants rated their motion control higher in INTENTION compared to EXTERNAL ($beta = 4.9, p < .0001$), and higher in INTENTION compared to AUGMENTED ($beta = 3, p < .0001$). Further, higher control was observed in AUGMENTED compared to EXTERNAL ($beta = 1.9, p = .006$).

\begin{figure*}
    \centering
    \includegraphics[width=\textwidth]{figures/eeg_results.pdf}
    \caption{(a) Grand average event-related potential (ERP) at electrode Cz of pre-movement EEG data epochs preceding the last second before a movement, idle epochs of the same trial are plotted alongside but originate from a different time window, see section~\ref{eeg_methods}; (b) Bottom: Slope features for both idle and pre-movement classes at electrode Cz. Top: Scalp maps of slope values for both classes and all channels. (c): Mean and per participant ROC curves. Dotted line at 15 \% false positive rate indicate the selected threshold for the real-time application.}
    \label{fig:EEG_results}
\end{figure*}

\subsection{Classifier Performance}
Visual inspection of the amplitudes at electrode Cz revealed an increase in the difference between \textit{pre-movement} and \textit{idle} data segments towards the onset of the finger movement, see figure~\ref{fig:EEG_results} a. The slope feature for the exemplary channel discriminates well between the two classes (${t_{(8)}} = 3.6, p = .003$), see figure ~\ref{fig:EEG_results} b bottom. The scalp maps in figure~\ref{fig:EEG_results} b top show the (color-coded) mean slope for each channel and each class. Central channels on the contralateral side to the moving finger on the right hand show a negative slope for the pre-movement class and a neutral slope for the idle class. Furthermore, differences in slope were also observed at frontal electrodes over the left hemisphere as well as parietal electrodes, were a positive slope manifested only for the idle class. 

The grid-search over channels resulted in the BCI leveraging on average 11.5 (SD = 3.7) channels. \textcolor{green}{Besides channels C3, C4, and Cz, that were always included, other common channels (retained for at least 3 participants) included FT9, AF3, F5, F7, F2, FT8, and AF7.} The classifier cross-validation resulted in a mean F1 score of .7 (SD = .03), see figure~\ref{fig:EEG_results} c. We set the detection threshold to 15 \% false positive rate and at that rate, observed a mean threshold of 57 \% (SD = .02). Hence, on average, the classifier switched on the EMS when it predicted class \textit{pre-movement} with 57 \% probability.
% TODO add percentage of trials the prototype was activated before touching the screen (SD = ?)

\subsection{Subjective Reports}
Six out of the eight participants made a reference to the source of the EMS in the AUGMENTED condition. In that subgroup, most participants wondered how the system worked, one remarked, ``I can't explain how it works technically.''. They found the stimulation to be sporadic, even during their own actions, leading to a belief that it was coincidental. As one participant put it, ``I have no idea what controlled the stimulation, I think it was coincidence when stimulation occurred.''. Furthermore, participants reported that they found the timing of the stimulation to be unpredictable, with one participant noting, ``Stimulation was random in time.''. Some perceived the stimulation as externally triggered, yet partially responsive to their choices, resulting in a sentiment captured by, ``I think that the stimulation in the third block was partly involuntary, but partly as if it was following my decision / choice.''. One participant briefly considered the involvement of their brain waves in controlling the stimulation, but expressing doubts about this possibility, stating, ``But I don't think that is the case.''.

Some participants reported that the stimulation ``sometimes overlapped with the movement, but not often,'' or, it ``came only rarely.'' Another one noted, ``In a few cases, the stimulation came when I had already started the movement.'' For several participants, this overlap between their actions and the system's response occurred infrequently, with another user noting, ``Once, in the millisecond range between my planned movement and its execution.'' However, for some participants, there were moments of near-perfect synchronization, as described by one participant, ``In 3 cases it happened that my intention to press and the impulse of the device happened simultaneously.'' Another one noted, ``Sometimes it really happened that they overlapped. So that I just started to move, and then the device activated,'' that user estimated an overlap of ``40 \%'' while another said [we worked] ``15 \% together.'' These accounts collectively highlight the varied and occasionally synchronous nature of the system's timing in relation to the users' intentions.

In terms of valence sentiments for the cases where the stimulation aligned with participants' intentions, some participants indicated that the experience was positive, for example, participants described the experience as ``weird but funny'', ``pleasant'' or ``helped me with the execution'' and ``it was more of a collective movement.'' One participant remarked ``then it was ok to experience the stimulation, but also not more than ok'' while another one noted ``It had a bit of thinking ahead to it''. Some noted that they experienced an increase in their physical strength, remarking, ``supported my strength'', ``made me type more firmly'' and ``my typing performance was increased.'' On the other hand, some participants had negative sentiments during these moments of aligned stimulation, remarking, ``felt like I was still in competition with the system'', ``it did not feel like a acting together'' and ``on a psychological level it was a loss of control.''







%%% Resources and Ideas
% and the agency condition by xx (CI: ). Subsequent non-parametric comparisons revealed the significance of differences only between  
% xx (p = ).
% x\



%\missingfigure{A: Boxplot of questionnaire scores of three Blocks (block 2 grey, because not intention given) - play around with visualization of words that were used to explain control rating. B. Boxplots intentional binding (maybe integrate A and B) C. more subjective evaluation of interview / maybe world cloud / maybe sentiment analysis) } 



% Did they get faster? Calculate time between hand motion onset and button press. If yes, evidence that there was acceptance for the help/augmentation the system provided

\section{Discussion}

% summary here! BLUF!

% Outlook Ideas
% - Strength of EMS could be set by the amplitude of the RP -> is there evidence for a correlation between amplitude and 'strength' of volitional action?
%[ - EEG Signal may be contaminated by eye movement, maybe a worthwhile thing to build this interaction relying solely on eye tracking information, gaze fixating on the target before moving]
%- not only physical integration but also other adpative interfaces, or control of mobile phone (see Cornelio2022 review); % say here somewhere what action augmentation is and how it differs from body and environment/outcome augmentation

% \section{Conclusions, Limitations and Opportunities}


% \section{Introduction}

% % First Paragraph
% % CORE MESSAGE OF THIS PARAGRAPH:
% \todo{P1.1. What is the large scope of the problem?}
% \todo{P1.2. What is the specific problem?}

% % Second Paragraph
% % CORE MESSAGE OF THIS PARAGRAPH:
% \todo{P2. The second paragraph should be about what have others been doing}
% \todo{P2.3. Why is the problem important? Why was this work carried out?}

% % Third Paragraph
% % CORE MESSAGE OF THIS PARAGRAPH:
% \todo{P3.4. What have you done?}
% \todo{P3.5. What is new about your work?}

% % Fourth paragraph
% % CORE MESSAGE OF THIS PARAGRAPH:
% \todo{P4.6. What did you find out? What are the concrete results?}
% \todo{P4.7. What are the implications? What does this mean for the bigger picture?}

% \section{Related Work}


% \section{Study Design}


% \section{Results}

% \section{Discussion}


% \section{Conclusion}


%%
%% The acknowledgments section is defined using the "acks" environment
%% (and NOT an unnumbered section). This ensures the proper
%% identification of the section in the article metadata, and the
%% consistent spelling of the heading.
\begin{acks}

% no section, just write down the acknowledgements as text
\end{acks}

%%
%% The next two lines define the bibliography style to be used, and
%% the bibliography file.
\bibliographystyle{ACM-Reference-Format}
\bibliography{input/paperpile}


%%
%% If your work has an appendix, this is the place to put it.
%\appendix
%\section{Research Methods}


\end{document}
\endinput
%%
%% End of file `sample-manuscript.tex'.
